\chapter{Adjungierte Funktoren erhalten Limiten}
Wir kommen nun zu einem der wichtigsten Aspekte von Adjunktionen: adjungierte
Funktoren erhalten (Ko)Limiten. Genauer:
% TODO ^

\begin{thSatz}[Adjungierte Funktoren erhalten (Ko)Limiten]
    \label{ch3:rapl}
    %
    Seien $\catA,\catB$ Kategorien und $F\colon\catA\to\catB,\;
    G\colon\catB\to\catA$ Funktoren mit $F\leftadjoint G$.
    Dann gilt:
    \begin{itemize}
        \item
            Der rechtsadjungierte Funktor~$G$ erhält Limiten.
        \item
            Der linksadjungierte Funktor~$F$ erhält Kolimiten.
    \end{itemize}
\end{thSatz}

Dies macht deutlich, wie nützlich es für uns ist, wenn wir wissen, dass ein
Funktor ein rechts- oder linksadjungierter Funktor ist. Insbesondere werden wir
dies im Folgenden auch an einigen Beispielen erkennen. Davon abgesehen liefert
\cref{ch3:rapl} aber auch \emph{das Kriterium} schlechthin, um zu zeigen, dass
ein Funktor keinen links- oder rechtsadjungierten Funktor besitzen kann:

\begin{thKorollar}[Nicht-Existenz adjungierter Funktoren]
    \label{ch3:raplkontra}
    %
    Sei $F\colon\catA\to\catB$ ein Funktor zwischen Kategorien $\catA,\catB$.
    Falls $F$ einen (bliebigen!) Limes (Kolimes) nicht erhält, 
    so besitzt $F$ keinen linksadjungierten (rechtsadjungierten) Funktor.
\end{thKorollar}

Das Korollar folgt sofort mit Kontraposition aus \cref{ch3:rapl}. Der Beweis des
Satzes ist mit unserem bisherigen Vorwissen nicht mehr schwer:

\begin{proof}[Beweis von \cref{ch3:rapl}]
    Sei $I$ eine kleine Kategorie und $D\colon I\to\catA$ ein $I$-Diagramm
    in $\catA$, für das der Limes $\lim_I D$ existiert. Dann gilt für alle
    Objekte $A\in\catA$:
    \begin{align*}
        \catA\bigl(A, G(\lim\nolimits_I D)\bigr)
        &\cong \catB(FA, \lim\nolimits_I D)          \\
        &\cong \lim\nolimits_I \catB(FA, D\blank)    \\
        &\cong \lim\nolimits_I \catB(A, GD\blank)    \\
        &\cong \catB(A,\lim\nolimits_I GD)
    \end{align*}
    Dabei haben wir zweimal die Isomorphie zwischen Morphismenmengen vermöge der
    Adjunktion $F\leftadjoint G$ ausgenutzt und zweimal, dass (kovariante) 
    Hom-Funktoren mit Limiten vertauschen. Das Yonea-Lemma liefert nun: 
    \[ G(\lim\nolimits_I D) \cong \lim\nolimits_I GD  . \]
    Der Fall für $F$ verläuft analog (unter Beachtung der Tatsache, dass der
    kontravariante Hom-Funktor Kolimiten auch zu Limiten macht).
    \\
\end{proof}

\begin{thBeispiel}[Vergissfunktoren vs. (Ko)Limiten]\hfill
    \begin{itemize}
        \item
            In \cref{ch1:bsp:TopVergiss} haben wir gesehen, dass der Vergissfunktor
            $U\colon\Top\to\Set$ sowohl einen links- als auch einen rechtsadjungierten
            Funktor besitzt. Nach \cref{ch3:rapl} erhält $U$ somit also alle(!) Limiten
            \emph{und} Kolimiten.
            
        \item
            Betrachten wir hingegen Vergissfunktoren von algebraischen Kategorien, so
            sehen wir, dass diese im Allgemeinen nicht ganz so gutartig sind. Nehmen wir
            beispielsweise den bei Loher\cite[1.3]{talk:loher} eingeführten
            Vergissfunktor $U\colon \Vect_k\to\Set$ von der Kategorie der Vektorräume
            über einem gegebenen Körper~$k$ nach $\Set$, so erhält dieser zwar alle
            Limiten (denn der freie Funktor $\Set\to\Vect_k$ ist ein zu $U$
            linksadjungierter Funktor), aber $U$ erhält nicht alle Kolimiten. Dies sehen
            wir zum Beispiel anhand des initialen Objekts $0\in\Vect_k$ (trivialer
            Vektorraum) ein, denn $U(0) = \{0\}$, aber eine einelementige Menge ist
            \emph{kein} initiales Objekt in $\Set$. Insbesondere wissen wir damit auch,
            dass $U$ tatsächlich überhaupt keinen rechtsadjungierten Funktor
            besitzen kann. \pcref{ch3:raplkontra}
    \end{itemize}
\end{thBeispiel}

\begin{thBeispiel}[Tensorprodukt vs. direkte Summe]
    \newcommand{\tensorM}{\blank\otimes M}
    %
    Sei $R$ ein kommutativer Ring und $M\in\Mod_R$ ein $R$-Modul. Wir betrachten
    den Funktor $\tensorM\colon\Mod_R\to\Mod_R$, von dem wir bereits
    wissen (siehe Loher\cite[2.8]{talk:loher}), dass er einen rechtsadjungierten
    Funktor besitzt (nämlich den Funktor $\operatorname{Hom}(M,\blank)\colon
    \Mod_R\to\Mod_R$). Also ist $\tensorM$ ein linksadjungierter Funktor und als
    solcher erhält $\tensorM$ alle Kolimiten. Da die direkte Summe von
    $R$-Moduln gerade das Koprodukt (und damit ein Kolimes) in $\Mod_R$ ist,
    erhalten wir also sofort die äußerst nützliche Tatsache:
    Tensorieren vertauscht mit direkten Summen.
    In Formeln bedeutet dies: Ist $(N_i)_{i\in J}$ eine Familie von $R$-Moduln,
    dann gilt
    \[ \Bigl(\mkern2mu \bigoplus_{i\in J} N_i \Bigr)\oplus M
        \cong \bigoplus_{i\in J} (N_i\otimes M)
    . \]
    Man kann dies natürlich auch konkret auf Elementen nachrechnen oder mehrfach
    die universellen Eigenschaften von $\oplus$ und $\otimes$ anwenden; jedoch
    ist der Beweis \enquote{$\tensorM$ ist ein linksadjungierter Funktor} doch
    deutlich kürzer und angenehmer.
\end{thBeispiel}
