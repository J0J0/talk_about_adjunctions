\chapter{Adjungierte Funktoren erhalten Limiten}
Wir kommen nun zu einem der wichtigsten Aspekte von Adjunktionen: adjungierte
Funktoren erhalten (Ko)Limiten. Genauer:
% TODO ^

\begin{thSatz}[Adjungierte Funktoren erhalten (Ko)Limiten]
    \label{ch3:rapl}
    %
    Seien $\catA,\catB$ Kategorien und $F\colon\catA\to\catB,\;
    G\colon\catB\to\catA$ Funktoren mit $F\leftadjoint G$.
    Dann gilt:
    \begin{itemize}
        \item
            Der rechtsadjungierte Funktor~$G$ erhält Limiten.
        \item
            Der linksadjungierte Funktor~$F$ erhält Kolimiten.
    \end{itemize}
\end{thSatz}

Dies macht deutlich, wie nützlich es für uns ist, wenn wir wissen, dass ein
Funktor ein rechts- oder linksadjungierter Funktor ist. Insbesondere werden wir
dies im Folgenden auch an einigen Beispielen erkennen. Davon abgesehen liefert
\cref{ch3:rapl} aber auch \emph{das Kriterium} schlechthin, um zu zeigen, dass
ein Funktor keinen links- oder rechtsadjungierten Funktor besitzen kann:

\begin{thKorollar}[Nicht-Existenz adjungierter Funktoren]
    \label{ch3:raplkontra}
    %
    Sei $F\colon\catA\to\catB$ ein Funktor zwischen Kategorien $\catA,\catB$.
    Falls $F$ einen (bliebigen!) Limes (Kolimes) nicht erhält, 
    so besitzt $F$ keinen linksadjungierten (rechtsadjungierten) Funktor.
\end{thKorollar}

Das Korollar folgt sofort mit Kontraposition aus \cref{ch3:rapl}. Für den Beweis
des Satzes benötigen wir noch ein Lemma, zu dessen Formulierung wir wiederum den
Diagonalfunktor~$\Delta$ brauchen:

\begin{thErinnerDef}[Diagonalfunktor]
    \label{ch3:def:diagonalfunktor}
    %
    Sei $I$ eine kleine Kategorie und $\catA$ eine Kategorie.
    \begin{itemize}
        \item 
            Für alle $A\in\catA$ bezeichne $\Delta^I_A$ das $I$-Diagramm in
            $\catA$, welches alle Objekte aus $I$ auf $A$ abbildet und alle
            Morphismen auf $\id_A$.
            
        \item
            Der \emph{Diagonalfunktor $\Delta^I\colon\catA\to\catA^I$} ist 
            gegeben durch
            \[ A\mapsto \Delta^I_A \]
            auf Objekten und auf Morphismen wie folgt: Ist $f\colon A\to B$
            ein Morphismus von Objekten $A,B$ in $\catA$, so ist $\Delta^I(f)$
            die natürliche Transformation $\Delta^I_A \nattrafoto \Delta^I_B$,
            die auf jedem Objekt aus~$I$ durch $f$ gegeben ist.
            
        \item
            Ist $D\colon I\to\catA$ ein $I$-Diagramm in $\catA$ und $X\in\catA$,
            so ist ein Kegel $\bigl( X \to D(i) \bigr)_{i\in I}$ nichts anderes
            als eine natürliche Transformation $\Delta^I_X \nattrafoto D$.
            Insbesondere entspricht $\catA^I(\Delta^I_X, D)$ gerade der Menge aller
            Kegel über $D$.
    \end{itemize}
\end{thErinnerDef}

\begin{thLemma}
    \label{ch3:limesviadiagrammkategorie}
    %
    Sei $I$ eine kleine Kategorie, $\catA$ eine Kategorie und $D\colon
    I\to\catA$ ein $I$-Diagramm in $\catA$. Sei $X\in\catA$ und gelte
    \[ \catA(A,X) \cong \catA^I(\Delta^I A, D) \]
    natürlich in $A\in\catA$. Dann ist $X$ ein Limes von $D$ über $I$.
\end{thLemma}

\begin{proof}
    Sei $\phi$ ein expliziter natürlicher Isomorphismus
    \[ \catA^I(\Delta^I\blank, D) \nattrafoto \catA(\blank, X)  \]
    von Funktoren $\catA\op\to\Set$ (welcher nach Voraussetzung existiert).
    Sei weiter 
    \[ p \defeq \phi_X^{-1}(\id_X)\colon \Delta^I_X\nattrafoto D  . \]
    Wir behaupten nun, dass $(X,p)$ ein Limes von $D$ über $I$ ist. Sei also
    $Y\in\catA$ und $q\colon\Delta^I_Y\nattrafoto D$. Dann liefert 
    $f \defeq \phi_Y(q)$ einen Morphismus $Y\to X$ und wir müssen nur noch
    zeigen, dass für alle $i\in I$ schon $p_i\after f = q_i$ gilt
    (\cref{ch3:fig:trafos}, links) und dass es keine weiteren Mor\-phi\-smen
    $Y\to X$ mit dieser Eigenschaft gibt. Weil $\phi$ eine natürliche
    Transformation ist, gilt $f^*\after \phi_X = \phi_Y \after (\Delta^I f)^*$
    (d.\,h. das rechte Diagramm in \cref{ch3:fig:trafos} kommutiert).
    Nach Definition gilt $p = \phi_X^{-1}(\id_X)$ und damit
    \[ (f^*\after \phi_X)(p) = f^*(\id_X) = f  . \]
    Damit erhalten wir:
    \begin{align*}
        \phi_Y(q) = f
        &= \bigl( \phi_Y \after (\Delta^I f)^* \bigr)(p)    \\
        &= \phi_Y\bigl( p\after (\Delta^I f) \bigr)
    . \end{align*}
    Da $\phi_Y$ bijektiv ist, folgt $q = p \after (\Delta^I f)$ und dies
    bedeutet komponentenweise nichts anderes als 
    $q_i = \bigl( p \after (\Delta^I f) \bigr)_i = p_i \after f$
    für alle $i\in I$. Nehmen wir nun an, $g\colon Y\to X$ ist ein zweiter
    Morphismus mit $q = p\after (\Delta^I g)$. Dann ist $\tilde q\defeq
    \phi_Y^{-1}(g)$ eine natürliche Transformation, die analog zur Berechnung
    bei $q$ die Eigenschaft $\tilde q = p \after (\Delta^I g)$ erfüllt. Dann
    folgt aber schon $q = \tilde q$ und damit $f = \phi_Y(q) = 
    \phi_Y(\tilde q) = g$. Dies zeigt die Behauptung.
    \\
    %
    \begin{figure}[b]
        \centering
        \begin{equation*}
            \begin{gathered}
                \xymatrix{
                    & \ar@/_1pc/[ddl]_{q_i} Y \ar@/^1pc/[ddr]^{q_j} 
                    \ar[d]^f &
                    \\
                    & \ar[dl]_(.4){p_i} X \ar[dr]^(.4){p_j} &
                    \\
                    D(i) & \cdots & D(j)
                }
            \end{gathered}
            %
            \hspace{2cm}
            %
            \begin{gathered}
                \xymatrix{
                    \catA^I(\Delta^I_X, D) \ar[r]^{\phi_X} \ar[d]^{(\Delta^I f)^*} &
                    \catA(X,X) \ar[d]_{f^*}
                    \\
                    \catA^I(\Delta^I_Y, D) \ar[r]_{\phi_Y} & \catA(Y,X)
                }
            \end{gathered}
        \end{equation*}
        \caption{Situation im Beweis von \cref{ch3:limesviadiagrammkategorie}}
        \label{ch3:fig:trafos}
    \end{figure}
\end{proof}

\begin{proof}[Beweis von \cref{ch3:rapl}]
    \belowpdfbookmark{Beweis von Satz~\ref{ch3:rapl}}{beweisrapl}
    %
    Sei $I$ eine kleine Kategorie und $D\colon I\to\catB$ ein $I$-Diagramm
    in $\catB$, für das der Limes $\lim_I D$ existiert. 
    Wir geben nun zuerst einen einfachen Beweis dafür, dass $G$ diesen Limes
    erhält, falls $\catA$ eine vollständige Kategorie\footnote{d.\,h. es
    existieren alle kleinen Limiten (also Limiten von Diagrammen über kleinen
    Kategorien)} ist. Es gilt für alle Objekte $A\in\catA$:
    \begin{align*}
        \catA\bigl(A, G(\lim\nolimits_I D)\bigr)
        &\cong \catB(FA, \lim\nolimits_I D)          \\
        &\cong \lim\nolimits_I \catB(FA, D\blank)    \\
        &\cong \lim\nolimits_I \catA(A, GD\blank)    \\
        &\cong \catA(A,\lim\nolimits_I GD)
    \end{align*}
    Dabei haben wir zweimal die (natürliche) Isomorphie zwischen
    Morphismenmengen vermöge der Adjunktion $F\leftadjoint G$ ausgenutzt und
    zweimal, dass (kovariante) Hom-Funktoren mit Limiten vertauschen. Das
    Yoneda-Lemma liefert nun: 
    \[ G(\lim\nolimits_I D) \cong \lim\nolimits_I GD  . \]
    Den Fall für $F$ erhält man dual (unter Beachtung der Tatsache, dass der
    kontravariante Hom-Funktor Kolimiten auch zu Limiten macht), wenn man
    fordert, dass $\catB$ kovollständig ist.
    
    \noindent
    Nun zum allgemeinen Fall.  Dann ist der kritische Punkt in obigem Beweis,
    dass wir nicht wissen, ob \enquote{$\lim_I GD$} überhaupt
    existiert.\footnote{%
        Interessanterweise findet man in vielen Quellen trotzdem den obigen
        Beweis ohne jegliche Einschränkungen an $\catA$ oder $\catB$. Falls
        jemand eine (einfache) Rechtfertigung dafür kennt, möge sich diese
        Person bitte bei mir melden \ldots%
    }
    Wie im ersten Fall erhalten wir für alle $A\in\catA$
    \[  \catA\bigl(A, G(\lim\nolimits_I D)\bigr)
        \cong \lim\nolimits_I \catA(A, GD\blank)
    \]
    natürlich in $A$. Nun macht man sich leicht durch die explizite Darstellung
    von Limiten in $\Set$ klar, dass der Limes auf der rechten Seite gerade die
    Menge aller natürlichen Transformationen $\Delta^I_A \nattrafoto GD$ ist.
    Das heißt für alle $A\in\catA$ gilt
    \[ \catA\bigl(A, G(\lim\nolimits_I D)\bigr)
        \cong \catA^I(\Delta^I A, GD)
    , \]
    natürlich in $A$. Mit \cref{ch3:limesviadiagrammkategorie} erhalten wir also
    wie gewünscht:
    \[ G(\lim\nolimits_I D) \cong \lim\nolimits_I GD  . \]
\end{proof}

\begin{thBeispiel}[Vergissfunktoren vs. (Ko)Limiten]\hfill
    \begin{itemize}
        \item
            In \cref{ch1:bsp:TopVergiss} haben wir gesehen, dass der Vergissfunktor
            $U\colon\Top\to\Set$ sowohl einen links- als auch einen rechtsadjungierten
            Funktor besitzt. Nach \cref{ch3:rapl} erhält $U$ somit also alle(!) Limiten
            \emph{und} Kolimiten.
            
        \item
            Betrachten wir hingegen Vergissfunktoren von algebraischen Kategorien, so
            sehen wir, dass diese im Allgemeinen nicht ganz so gutartig sind. Nehmen wir
            beispielsweise den bei Loher\cite[1.3]{talk:loher} eingeführten
            Vergissfunktor $U\colon \Vect_k\to\Set$ von der Kategorie der Vektorräume
            über einem gegebenen Körper~$k$ nach $\Set$, so erhält dieser zwar alle
            Limiten (denn der freie Funktor $\Set\to\Vect_k$ ist ein zu $U$
            linksadjungierter Funktor), aber $U$ erhält nicht alle Kolimiten. Dies sehen
            wir zum Beispiel anhand des initialen Objekts $0\in\Vect_k$ (trivialer
            Vektorraum) ein, denn $U(0) = \{0\}$, aber eine einelementige Menge ist
            \emph{kein} initiales Objekt in $\Set$. Insbesondere wissen wir damit auch,
            dass $U$ tatsächlich überhaupt keinen rechtsadjungierten Funktor
            besitzen kann. \pcref{ch3:raplkontra}
    \end{itemize}
\end{thBeispiel}

\begin{thBeispiel}[Tensorprodukt vs. direkte Summe und Quotienten]
    \label{ch3:bsp:tensorvskolimiten}
    \newcommand{\tensorM}{\blank\otimes M}
    %
    Sei $R$ ein kommutativer Ring und $M\in\Mod_R$ ein $R$-Modul. Wenn wir im
    Folgenden $\otimes$ schreiben, so meinen wir stets das Tensorprodukt über
    $R$. 
    
    \noindent
    Wir betrachten den Funktor $\tensorM\colon\Mod_R\to\Mod_R$, von dem wir
    bereits wissen (siehe Loher\cite[2.8]{talk:loher}), dass er einen
    rechtsadjungierten Funktor besitzt (nämlich den Funktor
    $\operatorname{Hom}(M,\blank)\colon \Mod_R\to\Mod_R$). Also ist $\tensorM$
    ein linksadjungierter Funktor und als solcher erhält $\tensorM$ alle
    Kolimiten. Da die direkte Summe von $R$-Moduln gerade das Koprodukt (und
    damit ein Kolimes) in $\Mod_R$ ist, erhalten wir also sofort die äußerst
    nützliche Tatsache: Tensorieren vertauscht mit direkten Summen.  In Formeln
    bedeutet dies: Ist $(N_i)_{i\in J}$ eine Familie von $R$-Moduln, dann gilt
    \[ \Bigl(\mkern2mu \bigoplus_{i\in J} N_i \Bigr)\otimes M
        \cong \bigoplus_{i\in J} (N_i\otimes M)
    . \]
    Man kann dies natürlich auch konkret auf Elementen nachrechnen oder mehrfach
    die universellen Eigenschaften von $\oplus$ und $\otimes$ anwenden; jedoch
    ist der Beweis \enquote{$\tensorM$ ist ein linksadjungierter Funktor} doch
    deutlich kürzer und angenehmer.
    
    \medskip\noindent
    Sei nun $\mf a\triangleleft R$ ein Ideal in $R$. Wie man sich leicht klarmacht,
    ist der Pushout des Diagramms 
    \[ 
        \begin{gathered}
            \xymatrix@=4mm{ \mf a \ar[d] \ar@{ `->}[r] & R \\ 0 & } 
        \end{gathered}
        \qquad\rightsquigarrow\qquad
        \begin{gathered}
            \xymatrix@=4mm{ \mf a \ar[d] \ar@{ `->}[r] & R \ar[d] \\ 
            0 \ar[r] & R/\mf a }
        \end{gathered}
    \]
    gerade der Quotient $R/\mf a$ (wobei dies allgemein für Pushouts in $\Mod_R$
    gilt). Also können wir \enquote{Quotientenbildung} als Kolimes auffassen und
    nachdem $\tensorM$ ein linksadjungierter Funktor ist, muss also auch
    \[ 
        \begin{gathered}
            \xymatrix@=4mm{ \mf a\otimes M \ar[d] \ar[r] & R\otimes M \ar[d]
                           \\ 0\otimes M \ar[r] & (R/\mf a) \otimes M }
        \end{gathered}
        \qquad\rightsquigarrow\qquad
        \begin{gathered}
            \xymatrix@=4mm{ \mf a\otimes M \ar[d] \ar[r] & M \ar[d]
                           \\ 0 \ar[r] & (R/\mf a) \otimes M }
        \end{gathered}
    \]
    ein Pushout-Diagramm in $\Mod_R$ sein, mit
    \[ (R/\mf a) \otimes M \;\cong\; \Quot{R\otimes M}{\mf a\otimes M} . \]
    Also müssen wir nur wissen, wie $\mf a\otimes M$ als als Teilmenge von
    $R\otimes M \cong M$ aussieht. Man sieht leicht, dass unter dem
    Iasomorphismus (gegeben auf elementaren Tensoren durch)
    \[ R\otimes M \to M, \quad r\otimes m \mapsto rm \]
    das Bild von $\mf a\otimes M$ gerade $\mf aM\subset M$ entspricht.
    (Achtung, im Allgemeinen gilt jedoch \emph{nicht}
    $\mf a\otimes M \cong \mf aM$. Letztere Isomorphie ist aber
    insbesondere dann gegeben, wenn $M$ flach ist.) Also haben wir gezeigt:
    \[ (R/\mf a) \otimes M \;\cong\; \Quot{M}{\mf aM} \mkern1mu. \]
\end{thBeispiel}

\begin{thBeispiel}[Tensorprodukt vs. Limiten]
    \label{ch3:bsp:tensorvslimiten}
    \newcommand{\tensorQ}{\blank\otimes\Q}
    In \cref{ch3:bsp:tensorvskolimiten} haben wir gesehen, dass sich
    Tensorprodukte mit Kolimiten vertragen. Nun wollen wir zeigen, dass
    $\blank\otimes M$ (mit den Bezeichnern aus dem vorherigen Beispiel)
    Limiten und insbesondere Produkte im Allgemeinen nicht erhält, d.\,h.,
    dass \enquote{Tensorieren} nicht mit Limiten/Produkten vertauscht.

    \noindent
    Für diejenigen Leser, denen die \emph{$p$-adischen ganzen Zahlen~$\Z_p$} 
    für eine Primzahl $p\in\Z$ als projektiver Limes $\varprojlim_{n\in\N}
    \txtZQuot{p^n}$ bekannt sind, nennen wir gleich das interessante Beispiel:
    \[ \thickmuskip=8mu
        \Z_p\underset{\scriptscriptstyle\Z}\otimes\Q \cong \Q_p \not\cong 0 
        \cong  \varprojlim_{n\in\N} \,\Bigl( \ZQuot{p^n} 
        \underset{\scriptscriptstyle\Z}\otimes \Q \Bigr)
    . \]

    \noindent
    Wir wollen dieses Phänomen nun ohne Zuhilfenahme von $p$-adischen Zahlen
    etwas näher beleuchten.  Dazu betrachten wir den $\Z$-Modul $\Q$ und den
    Funktor $\tensorQ$ (wobei wir stets über $\Z$ tensorieren). Sei $p\in\Z$
    eine Primzahl. Weiter fassen wir $\N$ (zunächst) als diskrete
    Kategorie\footnote{d.\,h. als Katgorie, die außer den Identitätsmorphismen
    keine weiteren besitzt} auf und definieren das $\N$-Diagramm~$Z$ in
    $\Mod_\Z$ durch
    \[ \N\to\Mod_\Z, \quad n\mapsto \ZQuot{p^n}  \]
    auf Objekten (-- da es keine nicht-trivalen Morphismen gibt und Identitäten
    auf Identitäten abgebildet werden müssen gibt es auf Morphismen nichts zu
    tun). Dann ist der Limes von $Z$ über $\N$ in $\Mod_\Z$ gerade das folgende
    Produkt in $\Mod_R$:
    \[ \prod_{n\in\N} \ZQuot{p^n} \;=\; \lim_\N Z  . \]
    Für alle $n\in\N$ ist $\txtZQuot{p^n}$ offenbar ein Torsionsmodul über $\Z$,
    also
    \[ \Bigl( \ZQuot{p^n} \Bigr) \otimes \Q  \;\cong\; 0  . \]
    Insbesondere gilt somit:
    \[ \thickmuskip=8mu
        0 \cong \prod_{n\in\N} 0 
        \cong \prod_{n\in\N} \Bigl( \ZQuot{p^n} \otimes\Q\Bigr)
        = \lim_\N \,\bigl((Z\blank)\otimes\Q\bigr)
    . \]
    Aber es gilt
    \[ \Bigl( \lim_\N Z \Bigr) \otimes \Q 
        \;\cong\; \bigl(\Z\setminus\{0\}\bigr)^{-1} 
        \Bigl( \prod_{n\in\N} \ZQuot{p^n} \Bigr)
        \;\not\cong\; 0  , 
    \]
    denn $([1])_{n\in\N}/1$ ist ein nicht-trivales Element in der Lokalisierung
    $(\Z\setminus\{0\})^{-1} \bigl( \prod_{n\in\N} \txtZQuot{p^n} \bigr)$.
    Wenn wir nun noch klären, wie $\Z_p$ aus dem Eingangsbeispiel genau
    definiert ist, so lassen sich diese Argumente leicht darauf übertragen, was
    an dieser Stelle dem Leser zur Übung überlassen sei.
    
    \begin{thErinnerDef}[Ordnungskategorie, projektiver/induktiver Limes]%
        \label{ch3:def:ordkatprojindlim}
        \hfill\\
        Sei $(N,\leq)$ eine partiell geordnete Menge und sei $\catA$ eine
        Kategorie.
        \begin{itemize}
            \item
                Die (kleine) Kategorie $\Ord{(N,\leq)}$ sei gegeben durch
                $\Ob(\Ord{(N,\leq)}) \defeq N$ und für $i,j\in N$ gebe es
                einen eindeutigen Morphismus $i\to j$, falls $i\leq j$
                gilt (ansonsten keinen); sind $i,j,k\in N$ und $i\to j,\; j\to
                k$ Morphismen, so sei $(j\to k)\after (i\to j) \defeq (i\to k)$.
                
            \item
                Sei $(N,\leq)$ zusätzlich eine gerichtete Menge\footnote{d.\,h.
                eine partiell geordnete Menge und für alle $i,j\in N$ gebe es
                ein $k\in N$ mit $i\leq k$ und $j\leq k$} und sei $I \defeq
                \Ord{(N,\leq)}$.
                
                Sei $D\colon I\op\to\catA$ ein $I\op$-Diagramm in $\catA$. 
                Falls $D$ einen Limes über $I\op$ in $\catA$ besitzt, so
                definieren wir den \emph{projektiven Limes von $D$ über $N$}
                durch
                \[ \varprojlim_{n\in N} D(n) \defeq \lim\nolimits_{I\op} D  . \]
                
                Sei $\bar D\colon I\to\catA$ ein $I$-Diagramm in $\catA$. 
                Falls $\bar D$ einen Kolimes über $I$ in $\catA$ besitzt, so
                definieren wir den \emph{induktiven Limes von $\bar D$ über $N$}
                durch
                \[ \varinjlim_{n\in N} \bar D(n) 
                    \defeq \colim\nolimits_I \bar D  
                . \]
                %
                Wir nennen $D$ auch ein \emph{projektives System über $N$
                in $\catA$} und $\bar D$ ein \emph{induktives System über
                $N$ in $\catA$}.
        \end{itemize}
    \end{thErinnerDef}
    
    \begin{thDef}[Vervollständigung von Ringen, $p$-adische (ganze) Zahlen]\hfill
        \begin{itemize}
            \item
                Sei $R$ ein kommutativer Ring und $\mf a\triangleleft R$ ein
                Ideal. Dann definiert
                \begin{align*}
                    \N \ni n   &\mapsto \Quot{R}{\mf a^n}  \\[3pt]
                    (n\leq m)  &\mapsto \bigl(\mkern2mu
                        \Quot{R}{\mf a^m}  \to \Quot{R}{\mf a^n}, \;\;
                                       [x] \mapsto [x]
                    \mkern1mu \bigr)
                \end{align*}
                ein projektives System in $\Ring$ über $(\N,\leq)$ (mit der
                üblichen (Total-)Ordnung \enquote{$\leq$} auf~$\N$). Wir
                bezeichnen den projektiven Limes
                \[ \varprojlim_{n\in\N} \, \Bigl( \Quot{R}{\mf a^n} \Bigr) \]
                als \emph{$\mf a$-adische Vervollständigung von $R$}.
                
            \item
                Sei $p\in\Z$ eine Primzahl und $\Spann{p}$ das von $p$ in $\Z$
                erzeugte Ideal. Dann sind die \emph{$p$-adischen ganzen
                Zahlen~$\Z_p$} definiert als die $\Spann{p}$-adische
                Vervollständigung von $\Z$:
                \[ \Z_p \defeq \varprojlim_{n\in\N} \, \Bigl( \ZQuot{p^n} \Bigr)
                . \]
                Den Quotientenkörper $\operatorname{Quot}(\Z_p)$ des
                Integritätsrings~$\Z_p$ nennen wir \emph{$p$-adische Zahlen} und
                wir bezeichnen diesen mit $\Q_p$.
        \end{itemize}
    \end{thDef}
    
    \begin{thBemerkung}\hfill
        \begin{itemize}
            \item
                In $\Set,\Group,\Ring,\Mod_R,\Vect_k$ (für einen kommutativen
                Ring~$R$ und einen Körper~$k$) existieren stets alle projektiven
                Limiten und diese lassen sich wie folgt explizit angeben:
                Sei $(N,\leq)$ eine gerichtete Menge und 
                \[ \bigl( (X_i)_{i\in N},
                \, (f_{ij}\colon X_j\to X_i)_{i,j\in N,\, i\leq j} \bigr) \]
                ein
                projektives System über $(N,\leq)$ in einer der genannten
                Kategorien.\footnote{%
                    Diese Daten (zusammen mit gewissen Forderungen an die
                    Morphismen $f_{ij}$) bestimmen offenbar ein
                    $\Ord{(N,\leq)}$-Diagramm im Sinne von
                    \cref{ch3:def:ordkatprojindlim}%
                } Dann gilt:
                \[ \varprojlim_{i\in N} X_i 
                    = \Bigl\{
                        (x_i)_{i\in N} \in \prod\nolimits_{i\in N} X_i
                        \cMid\Big \forall\,i,j\in N\colon\;
                        i \leq j \implies x_i = f_{ij}(x_j)
                    \Bigr\}
                \]
                (mit komponentenweisen Verknüpfungen).
                
            \item
                In \cref{ch3:bsp:tensorvslimiten} betrachten wir Limiten in
                $\Mod_\Z$, aber anhand der expliziten Darstellung wird klar,
                dass wir für $\varprojlim_{n\in\N} \txtZQuot{p^n}$ dieselbe
                unterliegende Menge in $\Ring$ und $\Mod_\Z$ erhalten (bzw.
                sogar dieselbe zugrunde liegende abelsche Gruppe) und diese 
                nur mit unterschiedlichen Verknüpfungen ausgestattet ist.
        \end{itemize}
    \end{thBemerkung}
\end{thBeispiel}

\begin{thErinnerung}[Limes als Funktor]
    Sei $\catA$ eine vollständige Kategorie und $I$ eine kleine Kategorie.
    Wir fixieren für alle $I$-Diagramme in $\catA$ einen fest gewählten Limes.
    Dann ist $\lim_I$ ein Funktor von der Diagrammkategorie~$\catA^I$ nach
    $\catA$. Sind $D$ und $\bar D$ zwei $I$-Diagramm in $\catA$ und
    $\eta\colon D\nattrafoto \bar D$ eine natürliche Transformation,
    so ist $\lim_I\eta$ der (eindeutige) Morphismus $\lim_I D\to\lim_I\bar D$,
    der das Diagramm in \cref{ch3:fig:limesalsfunktor} zum Kommutieren bringt.
    %
    \begin{figure}
        \centering
        \begin{equation*}
                \xymatrix{
                    &        \ar@/_1.5pc/[ddl]_{\eta_i\after p_i} 
                    \lim_I D \ar@/^1.5pc/[ddr]^{\eta_j\after p_j} 
                    \ar@{-->}[d]^(.57){\lim_I\eta} &
                    \\
                    & \ar[dl]_(.4){\bar p_i} \lim_I\bar D 
                      \ar[dr]^(.4){\bar p_j} &
                    \\
                    \bar D(i) \ar[rr]^{\bar D(h)}
                    \ar@{}@<-16pt>[rr]|-{i \xrightarrow[\text{in $I$}]{h} j}
                    & & \bar D(j)
                }
        \end{equation*}
        \caption{Limes als Funktor, wobei $\eta\colon D\nattrafoto\bar D$ eine
            natürliche Transformation ist und $(\bar p_i)_{i\in I}$ bzw.
            $(p_i)_{i\in I}$ die zu $\lim_I\bar D$ bzw. $\lim_I D$ gehörigen
            Morphismen sind.}
        \label{ch3:fig:limesalsfunktor}
    \end{figure}
\end{thErinnerung}

\begin{thBeispiel}[Limes als rechtsadjungierter Funktor]
    \label{ch3:bsp:limesrechtsadjungiert}
    %
    Sei $\catA$ eine vollständige Kategorie und $I$ eine kleine Kategorie.
    Wir fixieren für alle $I$-Diagramme in $\catA$ einen fest gewählten Limes.
    Dann ist $\lim_I$ rechtsadjungiert zum Diagonalfunktor~$\Delta^I$
    \pcref{ch3:def:diagonalfunktor}, d.\,h. es gilt
    \[ \Delta^I \leftadjoint \lim\nolimits_I  . \]
\end{thBeispiel}

\begin{proof}
    \let\origlim=\lim
    \renewcommand{\lim}{\origlim\nolimits}
    %
    \begin{figure}
        \centering
        \begin{equation*}
                \xymatrix{
                    & \ar@/_1.5pc/[ddl]_{\eta_i} 
                    A \ar@/^1.5pc/[ddr]^{\eta_j} 
                    \ar@{-->}[d]_(.57){\exists!}^(.57){g_D^A(\eta)} &
                    \\
                    & \ar[dl]_(.4){p_i} \lim_I D 
                      \ar[dr]^(.4){p_j} &
                    \\
                    \bar D(i) \ar[rr]^{D(h)}
                    \ar@{}@<-16pt>[rr]|-{i \xrightarrow[\text{in $I$}]{h} j}
                    & & \bar D(j)
                }
        \end{equation*}
        \caption{Zuordnung $\catA^I(\Delta^I A, D) \to \catA(A,\lim_I D)$ im
            Beweis von \cref{ch3:bsp:limesrechtsadjungiert}}
        \label{ch3:fig:nattrafozumorphismus}
    \end{figure}
    %
    Sei $A\in\catA$ und $D\in\catA^I$ mit Limes $\bigl(\lim_I D,(p_i)_{i\in I}\bigr)$.
    Wir überlegen uns zunächst, dass tatsächlich
    \[ \catA^I(\Delta^I A, D) \cong \catA(A,\lim_I D) \]
    gilt. Sei $\eta\in\catA^I(\Delta^I A, D)$, d.\,h. $\eta$ ist eine natürliche
    Transformation $\Delta^I_A\nattrafoto D$. Wir hatten uns schon überlegt,
    dass $(A,\eta)$ damit gerade ein Kegel über $D$ ist, so dass wir aus der
    universellen Eigenschaft des Limes einen eindeutigen Morphismus
    $g_D^A(\eta)\colon A\to \lim_I D$ bekommen (siehe
    \cref{ch3:fig:nattrafozumorphismus}). Ist umgekehrt $g\colon A\to \lim_I D$
    ein Morphismus in $\catA$, so definiert $(p_i\after g)_{i\in I}$ eine
    natürliche Transformation $\Delta^I_A \nattrafoto D$. Wie eben erhalten wir
    zu dieser einen eindeutigen Morphismus  $\tilde g\colon A\to\lim_I D$ und
    wegen der Eindeutigkeit in der universellen Eigenschaft des Limes muss dann
    schon $g=\tilde g$ gelten. Damit haben wir gezeigt, dass
    \[ \catA^I(\Delta^I A, D) \overset{g_D^A}\longto \catA(A,\lim_I D) \]
    eine Bijektion ist. Es bleibt also zu zeigen, dass dies auch natürlich in
    $A$ und $D$ passiert.\footnote{Man sieht leicht, dass es zu unserer
        Definition äquivalent ist, die Natürlichkeit in jeder Variable einzeln
        zu prüfen. (Siehe auch Youcis\cite{www:an:youcis:productcat2}.)%
    }
    Dies wollen wir im Folgenden einmal direkt nachrechnen.
    
    \begin{description}
        \item[Natürlichkeit in \boldmath$A$]\hfill\\
            Seien $B$ und $A$ Objekte in $\catA$, $f\in\catA(B,A)$ (Achtung,
            Kontravarianz) und $D\in\catA^I$ mit Limes $\bigl(\lim_I
            D,(p_i)_{i\in I}\bigr)$. Weil $D$ konstant ist, schreiben wir
            einfach $g^A \defeq g_D^A$ und $g^B \defeq g_D^B$.
            Wir haben zu zeigen, dass das linke Diagramm in
            \cref{ch3:fig:limrechtsadjdiagramme} kommutiert.  Sei dazu
            $\eta\in\catA^I(\Delta^I A, D)$. Dann gilt (als Morphismen
            $B\to\lim_I D$ in $\catA$)
            \begin{align*}
                \bigl(f^* \after g^A\bigr)(\eta)
                &= f^*\bigl( g^A(\eta) \bigr)
                 = g^A(\eta) \circ f
                \qquad\text{und}\\[0.5ex]
                \bigl(g^B \circ (\Delta^I f)^*\bigr)(\eta)
                &= g^B\bigl( \eta \after (\Delta^I f) \bigr)
            . \end{align*}
            Für alle $i\in I$ erhalten wir
            \begin{align*}
                p_i\after g^A(\eta)\after f 
                &= \eta_i\after f
                \qquad\text{und}\\[0.5ex]
                p_i\after \bigl(g^B\bigl( \eta \after (\Delta^I f) \bigr)\bigr)
                &= \bigl( \eta \after (\Delta^I f)\bigr)_i
                 = \eta_i \after f
            , \end{align*}
            weswegen nach der universellen Eigenschaft von $\lim_I D$
            schon 
            \[ \bigl(f^* \after g^A\bigr)(\eta) 
                = \bigl(g^B \circ (\Delta^I f)^*\bigr)(\eta)
            \]
            gelten muss. Da $\eta$ beliebig war, folgt
            \[ f^* \after g^A = g^B \circ (\Delta^I f)^* , \]
            und gerade das wollten wir zeigen.
            
        \item[Natürlichkeit in \boldmath$D$]\hfill\\
            Seien $D,\bar D\in\catA^I$, $\eta\in\catA^I(D,\bar D)$ und
            $A\in\catA$. Seien $\bigl(\lim_I D,(p_i)_{i\in I}\bigr)$ und 
            $\bigl(\lim_I \bar D,(\bar p_i)_{i\in I}\bigr)$ die Limiten von 
            $D$ bzw. $\bar D$ über $I$. Da hier $A$ konstant ist, schreiben wir
            nur $g_D \defeq g_D^A$ und $g_{\bar D} \defeq g_{\bar D}^A$.
            Nun haben wir zu zeigen, dass das rechte Diagramm in
            \cref{ch3:fig:limrechtsadjdiagramme} kommutiert.  Sei dazu
            $\alpha\in\catA^I(\Delta^I A, D)$. Dann gilt (als Morphismen
            $A\to\lim_I\bar D$ in $\catA$)
            \begin{align*}
                \bigl((\lim_I\eta)_\ast \after g_D \bigr)(\alpha)
                &= (\lim_I\eta)_\ast \bigl( g_D(\alpha) \bigr)
                 = (\lim_I\eta) \after g_D(\alpha)
                \qquad\text{und}\\[0.5ex]
                \bigl(g_{\bar D}\after\eta_\ast\bigr)(\alpha)
                &= g_{\bar D}\bigl( \eta_\ast(\alpha) \bigr)
                 = g_{\bar D}(\eta\after\alpha)
            . \end{align*}
            Für alle $i\in I$ erhalten wir
            \begin{align*}
                \bar p_i \after (\lim_I\eta) \after \bigl(g_D(\alpha)\bigr)
                &= \eta_i \after p_i \after \bigl(g_D(\alpha)\bigr)
                 = \eta_i \after \alpha_i
                \qquad\text{und}\\[0.5ex]
                \bar p_i \after \bigl(g_{\bar D}(\eta\after\alpha)\bigr)
                &= (\eta\after\alpha)_i = \eta_i\after\alpha_i
            , \end{align*}
            weswegen nach der universellen Eigenschaft von $\lim_I\bar D$
            schon
            \[ \bigl((\lim_I\eta)_\ast \after g_D \bigr)(\alpha)
                = \bigl(g_{\bar D}\after\eta_\ast\bigr)(\alpha)
            \]
            gelten muss. Da $\alpha$ beliebig war, folgt
            \[ (\lim_I\eta)_\ast \after g_D = g_{\bar D}\after\eta_\ast , \]
            und das wollten wir zeigen.
    \end{description}
    Also ist $g_D^A$ tatsächlich bijektiv und natürlich in $A$ und $D$ und damit
    ist alles gezeigt.
    \\
    %
    \begin{figure}
        \centering
        \begin{equation*}
            \begin{gathered}
                \xymatrix{
                    \catA^I(\Delta^I A, D) \ar[r]^{g^A} \ar[d]^{(\Delta^I f)^*} &
                    \catA(A, \lim_I D) \ar[d]_{f^*}
                    \\
                    \catA^I(\Delta^I B, D) \ar[r]_{g^B} & \catA(B, \lim_I D)
                }
            \end{gathered}
            %
            \hspace{2cm}
            %
            \begin{gathered}
                \xymatrix{
                    \catA^I(\Delta^I A, D) \ar[r]^{g_D} \ar[d]^{\eta_\ast} &
                    \catA(A, \lim_I D) \ar[d]_{(\lim_I\eta)_\ast}
                    \\
                    \catA^I(\Delta^I A, \bar D) \ar[r]_{g_{\bar D}} &
                    \catA(A, \lim_I\bar D)
                }
            \end{gathered}
        \end{equation*}
        \caption{Diagramme, deren Kommutativität im Beweis von
            \cref{ch3:bsp:limesrechtsadjungiert} zum Nachweis der Natürlichkeit
            gezeigt wird}
        \label{ch3:fig:limrechtsadjdiagramme}
    \end{figure}
\end{proof}

\begin{thBemerkung}\hfill
    \begin{itemize}
        \item
            Dual kann man zeigen, dass man \enquote{Kolimesbildung}
            als Funktor auffassen kann und dass dieser linksadjungiert
            zum Diagonalfunktor ist.
            
        \item
            Wir kennen nun einen zweiten Beweis, dass in vollständigen
            Kategorien Limiten mit Limiten vertauschen, denn wir haben
            soeben gezeigt, dass man \enquote{Limesbildung} als
            rechtsadjungierten Funktor auffasen kann und als solcher
            vertauscht dieser nach \cref{ch3:rapl} mit Limiten.
    \end{itemize}
\end{thBemerkung}
