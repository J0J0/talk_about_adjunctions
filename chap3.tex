\chapter{Adjungierte Funktoren erhalten Limiten}
Wir kommen nun zu einem der wichtigsten Aspekte von Adjunktionen: adjungierte
Funktoren erhalten (Ko)Limiten. Genauer:
% TODO ^

\begin{thSatz}[Adjungierte Funktoren erhalten (Ko)Limiten]
    \label{ch3:rapl}
    %
    Seien $\catA,\catB$ Kategorien und $F\colon\catA\to\catB,\;
    G\colon\catB\to\catA$ Funktoren mit $F\leftadjoint G$.
    Dann gilt:
    \begin{itemize}
        \item
            Der rechtsadjungierte Funktor~$G$ erhält Limiten.
        \item
            Der linksadjungierte Funktor~$F$ erhält Kolimiten.
    \end{itemize}
\end{thSatz}

Dies macht deutlich, wie nützlich es für uns ist, wenn wir wissen, dass ein
Funktor ein rechts- oder linksadjungierter Funktor ist. Insbesondere werden wir
dies im Folgenden auch an einigen Beispielen erkennen. Davon abgesehen liefert
\cref{ch3:rapl} aber auch \emph{das Kriterium} schlechthin, um zu zeigen, dass
ein Funktor keinen links- oder rechtsadjungierten Funktor besitzen kann:

\begin{thKorollar}[Nicht-Existenz adjungierter Funktoren]
    \label{ch3:raplkontra}
    %
    Sei $F\colon\catA\to\catB$ ein Funktor zwischen Kategorien $\catA,\catB$.
    Falls $F$ einen (bliebigen!) Limes (Kolimes) nicht erhält, 
    so besitzt $F$ keinen linksadjungierten (rechtsadjungierten) Funktor.
\end{thKorollar}

Das Korollar folgt sofort mit Kontraposition aus \cref{ch3:rapl}. Für den Beweis
des Satzes benötigen wir noch ein Lemma, zu dessen Formulierung wir wiederum den
Diagonalfunktor~$\Delta$ brauchen:

\begin{thErinnerDef}[Diagonalfunktor]
    Sei $I$ eine kleine Kategorie und $\catA$ eine Kategorie.
    \begin{itemize}
        \item 
            Für alle $A\in\catA$ bezeichne $\Delta^I_A$ das $I$-Diagramm in
            $\catA$, welches alle Objekte aus $I$ auf $A$ abbildet und alle
            Morphismen auf $\id_A$.
            
        \item
            Der \emph{Diagonalfunktor $\Delta^I\colon\catA\to\catA^I$} ist 
            gegeben durch
            \[ A\mapsto \Delta^I_A \]
            auf Objekten und auf Morphismen wie folgt: Ist $f\colon A\to B$
            ein Morphismus von Objekten $A,B$ in $\catA$, so ist $\Delta^I(f)$
            die natürliche Transformation $\Delta^I_A \nattrafoto \Delta^I_B$,
            die auf jedem Objekt aus~$I$ durch $f$ gegeben ist.
            
        \item
            Ist $D\colon I\to\catA$ ein $I$-Diagramm in $\catA$ und $X\in\catA$,
            so ist ein Kegel $\bigl( X \to D(i) \bigr)_{i\in I}$ nichts anderes
            als eine natürliche Transformation $\Delta^I_X \nattrafoto D$.
            Insbesondere entspricht $\catA^I(\Delta^I_X, D)$ gerade der Menge aller
            Kegel über $D$.
    \end{itemize}
\end{thErinnerDef}

\begin{thLemma}
    \label{ch3:limesviadiagrammkategorie}
    %
    Sei $I$ eine kleine Kategorie, $\catA$ eine Kategorie und $D\colon
    I\to\catA$ ein $I$-Diagramm in $\catA$. Sei $X\in\catA$ und gelte
    \[ \catA(A,X) \cong \catA^I(\Delta^I A, D) \]
    natürlich in $A\in\catA$. Dann ist $X$ ein Limes von $D$ über $I$.
\end{thLemma}

\begin{proof}
    Sei $\phi$ ein expliziter natürlicher Isomorphismus
    \[ \catA^I(\Delta^I\blank, D) \nattrafoto \catA(\blank, X)  \]
    von Funktoren $\catA\op\to\Set$ (welcher nach Voraussetzung existiert).
    Sei weiter 
    \[ p \defeq \phi_X^{-1}(\id_X)\colon \Delta^I_X\nattrafoto D  . \]
    Wir behaupten nun, dass $(X,p)$ ein Limes von $D$ über $I$ ist. Sei also
    $Y\in\catA$ und $q\colon\Delta^I_Y\nattrafoto D$. Dann liefert 
    $f \defeq \phi_Y(q)$ einen Morphismus $Y\to X$ und wir müssen nur noch
    zeigen, dass für alle $i\in I$ schon $p_i\after f = q_i$ gilt
    (\cref{ch3:fig:trafos}, links) und dass es keine weiteren Mor\-phi\-smen
    $Y\to X$ mit dieser Eigenschaft gibt. Weil $\phi$ eine natürliche
    Transformation ist, gilt $f^*\after \phi_X = \phi_Y \after (\Delta^I f)^*$
    (d.\,h. das rechte Diagramm in \cref{ch3:fig:trafos} kommutiert).
    Nach Definition gilt $p = \phi_X^{-1}(\id_X)$ und damit
    \[ (f^*\after \phi_X)(p) = f^*(\id_X) = f  . \]
    Damit erhalten wir:
    \begin{align*}
        \phi_Y(q) = f
        &= \bigl( \phi_Y \after (\Delta^I f)^* \bigr)(p)    \\
        &= \phi_Y\bigl( p\after (\Delta^I f) \bigr)
    . \end{align*}
    Da $\phi_Y$ bijektiv ist, folgt $q = p \after (\Delta^I f)$ und dies
    bedeutet komponentenweise nichts anderes als 
    $q_i = \bigl( p \after (\Delta^I f) \bigr)_i = p_i \after f$
    für alle $i\in I$. Nehmen wir nun an, $g\colon Y\to X$ ist ein zweiter
    Morphismus mit $q = p\after (\Delta^I g)$. Dann ist $\tilde q\defeq
    \phi_Y^{-1}(g)$ eine natürliche Transformation, die analog zur Berechnung
    bei $q$ die Eigenschaft $\tilde q = p \after (\Delta^I g)$ erfüllt. Dann
    folgt aber schon $q = \tilde q$ und damit $f = \phi_Y(q) = 
    \phi_Y(\tilde q) = g$. Dies zeigt die Behauptung.
    \\
    %
    \begin{figure}[b]
        \centering
        \begin{equation*}
            \begin{gathered}
                \xymatrix{
                    & \ar@/_1pc/[ddl]_{q_i} Y \ar@/^1pc/[ddr]^{q_j} 
                    \ar[d]^f &
                    \\
                    & \ar[dl]_(.4){p_i} X \ar[dr]^(.4){p_j} &
                    \\
                    D(i) & \cdots & D(j)
                }
            \end{gathered}
            %
            \hspace{2cm}
            %
            \begin{gathered}
                \xymatrix{
                    \catA^I(\Delta^I_X, D) \ar[r]^{\phi_X} \ar[d]^{(\Delta^I f)^*} &
                    \catA(X,X) \ar[d]_{f^*}
                    \\
                    \catA^I(\Delta^I_Y, D) \ar[r]_{\phi_Y} & \catA(Y,X)
                }
            \end{gathered}
        \end{equation*}
        \caption{Situation im Beweis von \cref{ch3:limesviadiagrammkategorie}}
        \label{ch3:fig:trafos}
    \end{figure}
\end{proof}

\begin{proof}[Beweis von \cref{ch3:rapl}]
    \belowpdfbookmark{Beweis von Satz~\ref{ch3:rapl}}{beweisrapl}
    %
    Sei $I$ eine kleine Kategorie und $D\colon I\to\catB$ ein $I$-Diagramm
    in $\catB$, für das der Limes $\lim_I D$ existiert. 
    Wir geben nun zuerst einen einfachen Beweis dafür, dass $G$ diesen Limes
    erhält, falls $\catA$ eine vollständige Kategorie\footnote{d.\,h. es
    existieren alle kleinen Limiten (also Limiten von Diagrammen über kleinen
    Kategorien)} ist. Es gilt für alle Objekte $A\in\catA$:
    \begin{align*}
        \catA\bigl(A, G(\lim\nolimits_I D)\bigr)
        &\cong \catB(FA, \lim\nolimits_I D)          \\
        &\cong \lim\nolimits_I \catB(FA, D\blank)    \\
        &\cong \lim\nolimits_I \catA(A, GD\blank)    \\
        &\cong \catA(A,\lim\nolimits_I GD)
    \end{align*}
    Dabei haben wir zweimal die (natürliche) Isomorphie zwischen
    Morphismenmengen vermöge der Adjunktion $F\leftadjoint G$ ausgenutzt und
    zweimal, dass (kovariante) Hom-Funktoren mit Limiten vertauschen. Das
    Yoneda-Lemma liefert nun: 
    \[ G(\lim\nolimits_I D) \cong \lim\nolimits_I GD  . \]
    Den Fall für $F$ erhält man dual (unter Beachtung der Tatsache, dass der
    kontravariante Hom-Funktor Kolimiten auch zu Limiten macht), wenn man
    fordert, dass $\catB$ kovollständig ist.
    
    \noindent
    Nun zum allgemeinen Fall.  Dann ist der kritische Punkt in obigem Beweis,
    dass wir nicht wissen, ob \enquote{$\lim_I GD$} überhaupt
    existiert.\footnote{%
        Interessanterweise findet man in vielen Quellen trotzdem den obigen
        Beweis ohne jegliche Einschränkungen an $\catA$ oder $\catB$. Falls
        jemand eine (einfache) Rechtfertigung dafür kennt, möge sich diese
        Person bitte bei mir melden \ldots%
    }
    Wie im ersten Fall erhalten wir für alle $A\in\catA$
    \[  \catA\bigl(A, G(\lim\nolimits_I D)\bigr)
        \cong \lim\nolimits_I \catA(A, GD\blank)
    \]
    natürlich in $A$. Nun macht man sich leicht durch die explizite Darstellung
    von Limiten in $\Set$ klar, dass der Limes auf der rechten Seite gerade die
    Menge aller natürlichen Transformationen $\Delta^I_A \nattrafoto GD$ ist.
    Das heißt für alle $A\in\catA$ gilt
    \[ \catA\bigl(A, G(\lim\nolimits_I D)\bigr)
        \cong \catA^I(\Delta^I A, GD)
    , \]
    natürlich in $A$. Mit \cref{ch3:limesviadiagrammkategorie} erhalten wir also
    wie gewünscht:
    \[ G(\lim\nolimits_I D) \cong \lim\nolimits_I GD  . \]
\end{proof}

\begin{thBeispiel}[Vergissfunktoren vs. (Ko)Limiten]\hfill
    \begin{itemize}
        \item
            In \cref{ch1:bsp:TopVergiss} haben wir gesehen, dass der Vergissfunktor
            $U\colon\Top\to\Set$ sowohl einen links- als auch einen rechtsadjungierten
            Funktor besitzt. Nach \cref{ch3:rapl} erhält $U$ somit also alle(!) Limiten
            \emph{und} Kolimiten.
            
        \item
            Betrachten wir hingegen Vergissfunktoren von algebraischen Kategorien, so
            sehen wir, dass diese im Allgemeinen nicht ganz so gutartig sind. Nehmen wir
            beispielsweise den bei Loher\cite[1.3]{talk:loher} eingeführten
            Vergissfunktor $U\colon \Vect_k\to\Set$ von der Kategorie der Vektorräume
            über einem gegebenen Körper~$k$ nach $\Set$, so erhält dieser zwar alle
            Limiten (denn der freie Funktor $\Set\to\Vect_k$ ist ein zu $U$
            linksadjungierter Funktor), aber $U$ erhält nicht alle Kolimiten. Dies sehen
            wir zum Beispiel anhand des initialen Objekts $0\in\Vect_k$ (trivialer
            Vektorraum) ein, denn $U(0) = \{0\}$, aber eine einelementige Menge ist
            \emph{kein} initiales Objekt in $\Set$. Insbesondere wissen wir damit auch,
            dass $U$ tatsächlich überhaupt keinen rechtsadjungierten Funktor
            besitzen kann. \pcref{ch3:raplkontra}
    \end{itemize}
\end{thBeispiel}

\begin{thBeispiel}[Tensorprodukt vs. direkte Summe und Quotienten]
    \newcommand{\tensorM}{\blank\otimes M}
    %
    Sei $R$ ein kommutativer Ring und $M\in\Mod_R$ ein $R$-Modul. Wir betrachten
    den Funktor $\tensorM\colon\Mod_R\to\Mod_R$, von dem wir bereits
    wissen (siehe Loher\cite[2.8]{talk:loher}), dass er einen rechtsadjungierten
    Funktor besitzt (nämlich den Funktor $\operatorname{Hom}(M,\blank)\colon
    \Mod_R\to\Mod_R$). Also ist $\tensorM$ ein linksadjungierter Funktor und als
    solcher erhält $\tensorM$ alle Kolimiten. Da die direkte Summe von
    $R$-Moduln gerade das Koprodukt (und damit ein Kolimes) in $\Mod_R$ ist,
    erhalten wir also sofort die äußerst nützliche Tatsache:
    Tensorieren vertauscht mit direkten Summen.
    In Formeln bedeutet dies: Ist $(N_i)_{i\in J}$ eine Familie von $R$-Moduln,
    dann gilt
    \[ \Bigl(\mkern2mu \bigoplus_{i\in J} N_i \Bigr)\otimes M
        \cong \bigoplus_{i\in J} (N_i\otimes M)
    . \]
    Man kann dies natürlich auch konkret auf Elementen nachrechnen oder mehrfach
    die universellen Eigenschaften von $\oplus$ und $\otimes$ anwenden; jedoch
    ist der Beweis \enquote{$\tensorM$ ist ein linksadjungierter Funktor} doch
    deutlich kürzer und angenehmer.
\end{thBeispiel}
