
\chapter{Vorwort und Notation}
Im letzten Vortrag (siehe Loher\cite{talk:loher}) haben wir \emph{Adjunktionen}
auf verschiedene Arten kennen gelernt und Beispiele gesehen. In diesem Skript
werden wir nun weitere Beispiele für Adjunktionen betrachten und darauf
eingehen, inwiefern man Adjunktionen miteinander verknüpfen kann. Im letzten
Kapitel werden wir beweisen, dass adjungierte Funktoren mit (Ko)Limiten
vertauschen, was die Nützlichkeit von Adjunktionen deutlich macht. Insbesondere
werden wir sehen, dass \enquote{Limesbilden} selbst ein rechtsadjungierter 
Funktor ist, womit wir erneut erkennen können, dass Limiten miteinander vertauschen.

\bigskip
Wir treffen folgende Vereinbarung:
Alle Kategorien seien \emph{lokal klein}, d.\,h. die Morphismen zwischen
zwei Objekten einer Kategorie bilden stets eine Menge (also ein Objekt
in $\Set$).

\bigskip
In diesem Skript werden folgende Notationen und Konventionen verwendet:
\begin{itemize}
    \item
        Ringe besitzen stets ein Einselement.
        
    \item
        Bekannte Kategorien (wobei wir nur die Objekte notieren):
        
        \hspace{6mm}
        \begin{tabular}{l@{\qquad}l}
            \emph{Bezeichnung} & \emph{Objekte}                         \\[2pt]
            $\Set$      &   Mengen                                      \\
            $\Top$      &   topologische Räume                          \\
            $\Ring$     &   Ringe                                       \\
            $\rMod R$   &   Moduln über einem kommutativen Ring~$R$     \\
            $\Cat$      &   kleine Kategorien
        \end{tabular}
        
    \item
        Weitere auftrende Kategorien werden mit $\cat A,\cat B, \cat C, \dots$
        bezeichnet. Sei $\catA$ eine Kategorie. Wir schreiben $A\in\catA$
        für ein Objekt~$A$ in $\catA$ und $\Ob(\catA)$ für die Klasse aller
        Objekte in $\catA$. Zu zwei Objekten $A,A'\in\catA$ bezeichne
        $\catA(A,A')$ die Menge aller Morphismen von $A$ nach $A'$.
        Anstatt von $f\in\catA(A,A')$ schreiben wir auch $f\colon A\to A'$ für
        einen Morphismus von $A$ nach $A'$.
        
    \item
        Eine \emph{kleine Kategorie} (d.\,h. eine Kategorie, für die die Klasse
        der Objekte eine Menge ist) wird meist mit $I$ bezeichnet. Einen Funktor
        $I\to\catA$ (von einer kleinen Kategorie~$I$ in eine weitere
        Kategorie~$\catA$) nenen wir auch \emph{$I$-Diagramm in $\catA$} und
        mit $\catA^I$ bezeichnen wir die \emph{Diagrammkategorie aller
        $I$-Diagramme in $\catA$} (mit natürlichen Transformationen als
        Morphismen).
        
    \item
        Sowohl $\subset$ als auch $\subseteq$ stehen für: enthalten oder gleich.
        Echt enthalten wird durch $\subsetneq$ gekennzeichnet.
        
    \item
        Die \emph{natürlichen Zahlen $\N$} beginnen mit $0$.
\end{itemize}
