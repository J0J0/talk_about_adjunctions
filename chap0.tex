
\chapter{Vorwort und Notation}
\ldots

\bigskip
Wir treffen folgende Vereinbarung:
Alle Kategorien seien \emph{lokal klein}, d.\,h. die Morphismen zwischen
zwei Objekten einer Kategorie bilden stets eine Menge (also ein Objekt
in $\Set$).

\bigskip
In diesem Skript wird folgende Notation verwendet:
\begin{itemize}
    \item
        Bekannte Kategorien (wobei wir nur die Objekte notieren):
        
        \hspace{6mm}
        \begin{tabular}{l@{\qquad}l}
            \emph{Bezeichnung} & \emph{Objekte}                         \\[2pt]
            $\Set$      &   Mengen                                      \\
            $\Top$      &   topologische Räume                          \\
            $\rMod R$   &   Moduln über einem kommutativen Ring~$R$      \\
            $\Cat$      &   kleine Kategorien
        \end{tabular}

    \item
        Weitere auftrende Kategorien werden mit $\cat A,\cat B, \cat C, \dots$
        bezeichnet. Sei $\catA$ eine Kategorie. Wir schreiben $A\in\catA$
        für ein Objekt~$A$ in $\catA$ und $\Ob(\catA)$ für die Klasse aller
        Objekte in $\catA$. Zu zwei Objekten $A,A'\in\catA$ bezeichne
        $\catA(A,A')$ die Menge aller Morphismen von $A$ nach $A'$.
        Anstatt von $f\in\catA(A,A')$ schreiben wir auch $f\colon A\to A'$ für
        einen Morphismus von $A$ nach $A'$.

    \item
        Sowohl $\subset$ als auch $\subseteq$ stehen für: enthalten oder gleich.
        Echt enthalten wird durch $\subsetneq$ gekennzeichnet.
    
    \item
        Die \emph{natürlichen Zahlen $\N$} beginnen mit $0$.
\end{itemize}
