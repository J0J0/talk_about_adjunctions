\chapter{Erinnerung und Beispiele}
Aus dem letzten Vortrag (siehe Loher\cite{talk:loher}) kennen wir das Konzept
\emph{adjungierter Funktoren}. Der Vollständigkeit halber besprechen wir noch
einmal knapp die verschiedenen Definitionsmöglichkeiten für Adjunktionen, welche
wir im Folgenden benutzen wollen.

\begin{thErinnerung}[Hom-Funktor(en)]
    Sei $\catA$ eine Kategorie. Dann ist $\catA(\blank,\blank)$ ein Funktor
    $\catA\op\times\catA\to\Set$, gegeben auf Objekten durch die offensichtliche
    Art und Weise und auf Morphismen wie folgt: seien $(A,B),(A',B')$ Objekte
    in $\catA\op\times\catA$ und sei $(f\op,g)\colon (A,B)\to (A',B')$ ein 
    Morphismus zwischen diesen, dann gilt:
    \[ \catA(f\op,g)\colon \catA(A,B)\to \catA(A',B'), \quad 
        h\mapsto g\after h\after f
    . \]
    Fixieren wir ein Objekt~$A\in\catA$, so ist
    also $\catA(A,\blank)$ ein kovarianter Funktor $\catA\to\Set$
    und $\catA(\blank,A)$ ein kontravarianter Funktor $\catA\to\Set$.
\end{thErinnerung}

\begin{thErinnerDef}[Adjungierte Funktoren, Adjunktion]
    Seien $F\colon \cat A\to\cat B$ und $G\colon \cat B\to \cat A$
    Funktoren zwischen Kategorien $\cat A, \cat B$. Wir nennen
    $F$ \emph{linksadjungiert zu $G$} und $G$ \emph{rechtsadjungiert zu $F$},
    wenn eine beiden folgenden äquivalenten Bedingungen erfüllt ist:

    \begin{description}
        \item[Morphismenmengen-Adjunktion]\hfill\\
            Es existiert ein natürlicher Isomorphismus 
            \[ \phi\colon \catB(F\blank,\blank)
                \nattrafoto \catA(\blank,G\blank)
            \]
            zwischen den Funktoren
            \begin{align*}
                \catB(\blank,\blank)\after(F\op\times\Id_\catB)&\colon
                \catA\op\times\catB\to\Set 
                \qquad\text{und}\\
                \catA(\blank,\blank)\after(\Id_\catA\op\times G)&\colon
                \catA\op\times\catB\to\Set
            . \end{align*}
            Wir beschreiben diese Situation auch wie folgt: Für alle Objekte
            $A\in\catA,\,B\in\catB$ ist
            \[ \phi_{A,B}\colon \catB(FA,B)\isorightarrow\catA(A,GB) \] 
            eine Bijektion und diese ist \emph{natürlich in $A$ und $B$}.

        \item[Einheit-Koeinheit-Adjunktion]\hfill\\
            Es existieren natürliche Transformationen
            \[  \eta\colon \Id_\catA \nattrafoto GF  \qundq
                \epsilon\colon FG \nattrafoto \Id_\catB 
             \]
            mit $\epsilon F \after F\eta = \id_F$ und 
            $G\epsilon \after \eta G = \id_G$ (\emph{Dreiecksidentitäten}, siehe
            \cref{ch1:fig:dreiecksid}). Wir nennen dann $\eta$ die
            \emph{Einheit} und $\epsilon$ die \emph{Koeinheit} der Adjunktion.
            %
            \begin{figure}
                \begin{equation*}
                    \begin{gathered} % treats vertical alignment
                        \xymatrix{
                            F \ar[r]^-{F\eta} \ar[dr]_-{\id_F} 
                            & FGF \ar[d]^(.43){\epsilon F}
                            \\ & F
                        }
                    \end{gathered}
                    \hspace{1.2cm}
                    \begin{gathered}
                        \xymatrix{
                            G \ar[r]^-{\eta G} \ar[dr]_-{\id_G} 
                            & GFG \ar[d]^(.43){G\epsilon}
                            \\ & G
                        }
                    \end{gathered}
                \end{equation*}
                \caption{Dreiecksidentitäten für Einheit und Koeinheit}
                \label{ch1:fig:dreiecksid}
            \end{figure}
    \end{description}
    
    \noindent
    Wir nennen
    \begin{itemize}
        \item 
            im ersteren Fall ein Tripel $(F,G,\phi)$,
        \item
            im zweiteren Fall ein Tupel $(F,G,\eta,\epsilon)$
    \end{itemize}
    eine \emph{Adjunktion}. In beiden Fällen schreiben wir $F\leftadjoint G$.
\end{thErinnerDef}

Einen Beweis für die Äquivalenz der Bedingungen findet man bei
Loher\cite{talk:loher} oder in einem beliebigen Buch zur Kategorientheorie.
Auch einige Beispiele wie  
\enquote{Frei $\leftadjoint$ Vergiss}, \enquote{Produkt $\leftadjoint$ Exponential}
oder \enquote{Tensor $\leftadjoint$ Hom} haben wir schon bei 
Loher\cite[1.3,\;1.4,\;2.8]{talk:loher} gesehen. Wir betrachten zunächst weitere
Beispiele, bevor wir allgemeine Eigenschaften von Adjunktionen näher
untersuchen.

\begin{thBeispiel}[Vergissfunktor auf \texorpdfstring{$\Top$}{Top}]
    Sei $U\colon\Top\to\Set$ der Vergissfunktor, der einem topologischen Raum
    seine unterliegende Menge zuordnet. Seien weiter $D,K\colon\Set\to\Top$ die
    Funktoren, die eine Menge mit der diskreten bzw. der
    Klumpentopologie\footnote{auch indiskrete oder chaotische Topologie genannt}
    ausstatten und so zu einem topologischen Raum machen. Alle drei Funktoren
    bilden Morphismen auf sich selbst\footnote{streng genommen: auf dieselbe
    Abbildung als Morphismus in der jeweils anderen Kategorie} ab.
    Sei $(Y,T)$ ein topologischer Raum und seien $(X,T_D)$ bzw. $(Z,T_K)$
    topologische Räume mit der diskreten bzw. der Klumpentopologie. Dann gilt
    \[  \Top\bigl( (X,T_D), (Y,T) \bigr) = \Set(X,Y)  \qundq
        \Top\bigl( (Y,T), (Z,T_D) \bigr) = \Set(X,Y)
    , \]
    denn jede Abbildung aus einem Raum mit der diskreten Topologie und jede
    Abbildung in einen Raum mit der Klumpentopologie ist stetig. Dies zeigt,
    dass die Funktoren $D,K$ auf Morphismen wohldefiniert sind. Wir behaupten
    nun, dass wir eine Kette von Adjunktionen $D\leftadjoint U\leftadjoint K$
    haben. Dies lässt leicht überprüfen, denn es gilt für alle Mengen~$Y$ und
    alle topologischen Räume $(X,T)$:
    \begin{alignat*}{2}
        \Top\bigl( DY, (X,T) \bigr) &= \Set(Y,X) &&= \Set\bigl( Y, U(X,T)\bigr)
        \quad\text{und}\\
        \Set\bigl( U(X,T), Y \bigr) &= \Set(X,Y) &&= \Top\bigl( (X,T), KY \bigr)
    . \end{alignat*}
    Also liefern $\phi^D_{Y,(X,T)} \defeq \id_{\Set(Y,X)}$ und 
    $\phi^K_{(X,T),Y} \defeq \id_{\Set(X,Y)}$ Adjunktionen $(D,U,\phi^D)$
    und $(U,K,\phi^K)$ (-- die Natürlichkeit ist klar).
\end{thBeispiel}
