\chapter{Erinnerung und Beispiele}
Aus dem letzten Vortrag (siehe Loher\cite{talk:loher}) kennen wir das Konzept
\emph{adjungierter Funktoren}. Der Vollständigkeit halber besprechen wir noch
einmal knapp die verschiedenen Definitionsmöglichkeiten für Adjunktionen, welche
wir im Folgenden benutzen wollen.

\begin{thErinnerung}[Hom-Funktor(en)]
    Sei $\catA$ eine Kategorie. Dann ist $\catA(\blank,\blank)$ ein Funktor
    $\catA\op\times\catA\to\Set$, gegeben auf Objekten durch die offensichtliche
    Art und Weise und auf Morphismen wie folgt: seien $(A,B),(A',B')$ Objekte
    in $\catA\op\times\catA$ und sei $(f\op,g)\colon (A,B)\to (A',B')$ ein 
    Morphismus zwischen diesen, dann gilt:
    \[ \catA(f\op,g)\colon \catA(A,B)\to \catA(A',B'), \quad 
        h\mapsto g\after h\after f
    . \]
    Fixieren wir ein Objekt~$A\in\catA$, so ist
    also $\catA(A,\blank)$ ein kovarianter Funktor $\catA\to\Set$
    und $\catA(\blank,A)$ ein kontravarianter Funktor $\catA\to\Set$.
\end{thErinnerung}

\begin{thErinnerDef}[Adjungierte Funktoren, Adjunktion]
    \label{ch1:def:adjunktion}
    %
    Seien $F\colon \cat A\to\cat B$ und $G\colon \cat B\to \cat A$
    Funktoren zwischen Kategorien $\cat A, \cat B$. Wir nennen
    $F$ \emph{linksadjungiert zu $G$} und $G$ \emph{rechtsadjungiert zu $F$},
    wenn eine beiden folgenden äquivalenten Bedingungen erfüllt ist:

    \begin{description}
        \item[Morphismenmengen-Adjunktion]\hfill\\
            Es existiert ein natürlicher Isomorphismus 
            \[ \phi\colon \catB(F\blank,\blank)
                \nattrafoto \catA(\blank,G\blank)
            \]
            zwischen den Funktoren
            \begin{align*}
                \catB(\blank,\blank)\after(F\op\times\Id_\catB)&\colon
                \catA\op\times\catB\to\Set 
                \qquad\text{und}\\
                \catA(\blank,\blank)\after(\Id_\catA\op\times G)&\colon
                \catA\op\times\catB\to\Set
            . \end{align*}
            Wir beschreiben diese Situation auch wie folgt: Für alle Objekte
            $A\in\catA,\,B\in\catB$ ist
            \[ \phi_{A,B}\colon \catB(FA,B)\isorightarrow\catA(A,GB) \] 
            eine Bijektion und diese ist \emph{natürlich in $A$ und $B$}.

        \item[Einheit-Koeinheit-Adjunktion]\hfill\\
            Es existieren natürliche Transformationen
            \[  \eta\colon \Id_\catA \nattrafoto GF  \qundq
                \epsilon\colon FG \nattrafoto \Id_\catB 
             \]
            mit $\epsilon F \after F\eta = \id_F$ und 
            $G\epsilon \after \eta G = \id_G$ (\emph{Dreiecksidentitäten}, siehe
            \cref{ch1:fig:dreiecksid}). Wir nennen dann $\eta$ die
            \emph{Einheit} und $\epsilon$ die \emph{Koeinheit} der Adjunktion.
            %
            \begin{figure}
                \begin{equation*}
                    \begin{gathered} % treats vertical alignment
                        \xymatrix{
                            F \ar[r]^-{F\eta} \ar[dr]_-{\id_F} 
                            & FGF \ar[d]^(.43){\epsilon F}
                            \\ & F
                        }
                    \end{gathered}
                    \hspace{1.2cm}
                    \begin{gathered}
                        \xymatrix{
                            G \ar[r]^-{\eta G} \ar[dr]_-{\id_G} 
                            & GFG \ar[d]^(.43){G\epsilon}
                            \\ & G
                        }
                    \end{gathered}
                \end{equation*}
                \caption{Dreiecksidentitäten für Einheit und Koeinheit}
                \label{ch1:fig:dreiecksid}
            \end{figure}
    \end{description}
    
    \noindent
    Wir nennen
    \begin{itemize}
        \item 
            im ersteren Fall ein Tripel $(F,G,\phi)$,
        \item
            im zweiteren Fall ein Tupel $(F,G,\eta,\epsilon)$
    \end{itemize}
    eine \emph{Adjunktion}. In beiden Fällen schreiben wir $F\leftadjoint G$.
\end{thErinnerDef}

Einen Beweis für die Äquivalenz der Bedingungen findet man bei
Loher\cite{talk:loher} oder in einem beliebigen Buch zur Kategorientheorie.
Auch einige Beispiele wie  
\enquote{Frei $\leftadjoint$ Vergiss}, \enquote{Produkt $\leftadjoint$ Exponential}
oder \enquote{Tensor $\leftadjoint$ Hom} haben wir schon bei 
Loher\cite[1.3,\;1.4,\;2.8]{talk:loher} gesehen. Wir betrachten zunächst weitere
Beispiele, bevor wir allgemeine Eigenschaften von Adjunktionen näher
untersuchen.

\begin{thBeispiel}[Vergissfunktor auf \texorpdfstring{$\Top$}{Top}]
    Sei $U\colon\Top\to\Set$ der Vergissfunktor, der einem topologischen Raum
    seine unterliegende Menge zuordnet. Seien weiter $D,K\colon\Set\to\Top$ die
    Funktoren, die eine Menge mit der diskreten bzw. der
    Klumpentopologie\footnote{auch indiskrete oder chaotische Topologie genannt}
    ausstatten und so zu einem topologischen Raum machen. Alle drei Funktoren
    bilden Morphismen auf sich selbst\footnote{streng genommen: auf dieselbe
    Abbildung als Morphismus in der jeweils anderen Kategorie} ab.
    Sei $(Y,T)$ ein topologischer Raum und seien $(X,T_D)$ bzw. $(Z,T_K)$
    topologische Räume mit der diskreten bzw. der Klumpentopologie. Dann gilt
    \[  \Top\bigl( (X,T_D), (Y,T) \bigr) = \Set(X,Y)  \qundq
        \Top\bigl( (Y,T), (Z,T_D) \bigr) = \Set(X,Y)
    , \]
    denn jede Abbildung aus einem Raum mit der diskreten Topologie und jede
    Abbildung in einen Raum mit der Klumpentopologie ist stetig. Dies zeigt,
    dass die Funktoren $D,K$ auf Morphismen wohldefiniert sind. Wir behaupten
    nun, dass wir eine Kette von Adjunktionen $D\leftadjoint U\leftadjoint K$
    haben. Dies lässt leicht überprüfen, denn es gilt für alle Mengen~$Y$ und
    alle topologischen Räume $(X,T)$:
    \begin{alignat*}{2}
        \Top\bigl( DY, (X,T) \bigr) &= \Set(Y,X) &&= \Set\bigl( Y, U(X,T)\bigr)
        \quad\text{und}\\
        \Set\bigl( U(X,T), Y \bigr) &= \Set(X,Y) &&= \Top\bigl( (X,T), KY \bigr)
    . \end{alignat*}
    Also liefern $\phi^D_{Y,(X,T)} \defeq \id_{\Set(Y,X)}$ und 
    $\phi^K_{(X,T),Y} \defeq \id_{\Set(X,Y)}$ Adjunktionen $(D,U,\phi^D)$
    und $(U,K,\phi^K)$ (-- die Natürlichkeit ist klar).
\end{thBeispiel}

\pagebreak[2]
\begin{thBeispiel}[Initiale und terminale Objekte]
    Sei $\catA$ eine Kategorie und sei $\ast$ das einzige Objekt der
    Kategorie~$\One$.\footnote{Kategorie mit genau einem Objekt und genau einem
    Morphismus} Sei weiter $C$ der eindeutig bestimmte Funktor $\catA\to\One$.
    Dann besitzt $\catA$ genau dann ein terminales Objekt, wenn es einen
    zu $C$ rechtsadjungierten Funktor gibt, und $\catA$ besitzt genau dann ein
    initiales Objekt, wenn es einen zu $C$ linksadjungierten Funktor gibt.

    \newcommand{\cA}{\circledchar[black!40]{$A$}}
    \newcommand{\cX}{\circledchar[black!40]{$X$}}
    \newcommand{\cY}{\circledchar[black!40]{$Y$}}
    Wir bemerken zunächst, dass ein Funktor $\One\to\catA$ nichts weiter ist
    als ein Objekt in $\catA$. Daher schreiben wir im Folgenden einfach
    $\cA$ für den Funktor $\One\to\catA$, der $\ast$ auf das Objekt~$A$ aus
    $\catA$ abbildet.
    
    Sei $\cX\colon\One\to\catA$ ein Funktor mit $C\leftadjoint\cX$. Ist dann
    $A\in\catA$ ein Objekt aus $\catA$, so gilt:
    \[ \One(\ast,\ast) = \One(C\!A,\ast) \cong \catA(A,\cX\ast) = \catA(A,X) . \]
    Da $\One(\ast,\ast) = \{\id_\ast\}$ genau aus dem Identitätsmorphismus auf
    $\ast$ besteht, muss es also genau ein Element in $\catA(A,X)$ geben, d.\,h.
    es gibt genau einen Morphismus $A\to X$. Da $A\in\catA$ beliebig war, ist
    $X$ also ein terminales Objekt. Analog erhalten wir für einen Funktor
    $\cY\colon\One\to\catA$ mit $\cY\leftadjoint C$ für alle $A\in\catA$:
    \[ \catA(Y,A) = \catA(\cY\ast,A) \cong \One(\ast,C\!A) = \One(\ast,\ast) . \]
    Damit ist $Y$ ein initiales Objekt in $\catA$.

    Sei nun umgekehrt $X$ ein terminales Objekt in $\catA$. Dann gilt
    $C\leftadjoint\cX$ vermöge der Adjunktion
    $\bigl(C,\cX, \eta, \epsilon\bigr)$ mit 
    \[ \eta \defeq (A{\to}X)_{A\in\catA} \qqundqq \epsilon \defeq (\id_\ast)_\ast
    . \]
    Die Natürlichkeit von $\epsilon$ ist klar und die von $\eta$ prüfen wir
    schnell nach: Sei $f\colon A\to A'$ ein Morphismus in $\catA$. Dann müssen
    wir zeigen, dass das Diagramm 
    \vspace{-2mm}
    \[
        \xymatrix@=3.5mm{A \ar[r]^f \ar[d] & A' \ar[d] \\ X \ar[r] & X}
    \]
    kommutiert. Aber da $X$ ein terminales Objekt ist, bleibt diesem Diagramm
    gar nichts anderes übrig, als zu kommutieren. Wir rechnen außerdem nach,
    dass $\eta$ und $\epsilon$ die Dreiecksidentitäten erfüllen. 
    Sei $A\in\catA$. Dann gilt
    \[ (\epsilon C \after C\eta)_A 
        = \epsilon_\ast \after C(A{\to}X)
        = \id_\ast \after \id_\ast
        = \id_\ast = \id_{C\!A}
    \]
    sowie
    \[ (\cX\mkern1mu\epsilon \after \eta\mkern1mu\cX)_\ast
        = \cX(\id_\ast) \after \eta_X
        = \id_X \after (X{\to}X)
        = \id_X = \id_{\cX\ast}
    . \]
    Völlig analog zeigt man, dass für ein initiales Objekt $Y\in\catA$ das Tupel
    \[ \bigl( \cY,\; C,\; (\id_\ast)_\ast,\; (Y{\to}A)_{A\in\catA} \bigr) \]
    eine Adjunktion $\cY\leftadjoint C$ definiert.
\end{thBeispiel}
