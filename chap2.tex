\chapter{Komposition von Adjunktionen}
Wir wollen nun untersuchen, wie wir Adjunktionen verknüpfen können. Einerseits
ist es genrell praktisch zu wissen, dass für Kategorien $\catA,\catB,\catC$ und
Funktoren
\vspace{-1mm}
\[ \tag{$\star$} \label{ch2:situation}
    \xymatrix{ \catA \ar@<3.5pt>[r]^F & \ar@<2pt>[l]^{G\vphantom{\bar G}} \catB 
    \ar@<3.5pt>[r]^{\bar F} & \ar@<2pt>[l]^{\bar G} \catC
    }
    \qquad\text{mit}\quad 
    F\leftadjoint G \text{\; und \,} \bar F\leftadjoint\bar G
\]
auch eine Adjunktion der komponierten Funktoren existiert. Andererseits
ermöglicht uns diese Feststellung die Betrachtung der Kategorie $\Adj$, wobei
$\Ob(\Adj)\defeq\Ob(\Cat)$ und $\Adj(\catA,\catB)$ definiert ist als die 
Menge aller Adjunktionen % TODO: check
zwischen den kleinen Kategorien $\catA$ und $\catB$;
die Verknüpfung von Morphismen ist die Komposition von Adjunktionen, die wir 
gleich besprechen werden, und die naheliegenden Adjunktion des
Identitätsfunktors zu sich selbst ist der Identitäsmorphimus auf einer kleinen
Kategorie. Außerdem gibt es eine zusätzliche Struktur auf $\Adj$, die diese
Kategorie zu einer sogenannten $2$-Kategorie macht. Dies geht jedoch über den
Inhalt dieses Skripts hinaus, so dass wir an dieser Stelle auf 
Mac~Lane\cite[\S\,IV.8,\;\S\,XII.3]{bookc:maclane97} und 
nLab\cite{www:nlab:2category} verweisen.

\begin{thProposition}[Komposition von Adjunktionen]
    Seien $\catA,\catB$ und $\catC$ Kategorien und seien $F,G,\bar F,\bar G$
    Funktoren wie in \eqref{ch2:situation}.
    Dann gilt auch $\bar F F \leftadjoint G\bar G$.
    %
    Genauer gilt:
    \begin{itemize}
        \item
            Sind $(F,G,\phi)$ und $(\bar F,\bar G,\bar\phi)$ zugehörige
            Adjunktionen zu obiger Situation, so ist auch
            \[ (\bar F F, G\bar G, \tilde\phi) \]  
            eine Adjunktion, wobei $\tilde\phi$ für alle 
            $A\in\catA,\; C\in\catC$ gegeben ist durch
            \[ \tilde\phi_{A,C} \defeq 
                \phi_{A,\bar GC} \after \bar\phi_{F\!A,C}
            . \]

        \item
            Sind $(F,G,\eta,\epsilon)$ und $(\bar F, \bar G, \bar\eta,
            \bar\epsilon)$ zugehörige Adjunktionen zu obiger Situation, so ist
            auch
            \[ (\bar F F, G\bar G, \tilde\eta, \tilde\epsilon) \]
            eine Adjunktion, wobei $\tilde\eta$ und $\tilde\epsilon$ gegeben
            sind durch
            \[ \tilde\eta \defeq \bar G\eta\bar F \after \bar\eta
                \qqundqq
                \tilde\epsilon \defeq \epsilon \after F\bar\epsilon G
            . \]
    \end{itemize}
\end{thProposition}
