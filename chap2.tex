\chapter{Komposition von Adjunktionen}
Wir wollen nun untersuchen, wie wir Adjunktionen verknüpfen können. Einerseits
ist es genrell praktisch zu wissen, dass für Kategorien $\catA,\catB,\catC$ und
Funktoren
\vspace{-1mm}
\[ \tag{$\star$} \label{ch2:situation}
    \xymatrix{ \catA \ar@<3.5pt>[r]^F & \ar@<2pt>[l]^{G\vphantom{\bar G}} \catB 
    \ar@<3.5pt>[r]^{\bar F} & \ar@<2pt>[l]^{\bar G} \catC
    }
    \qquad\text{mit}\quad 
    F\leftadjoint G \text{\; und \,} \bar F\leftadjoint\bar G
\]
auch eine Adjunktion der komponierten Funktoren existiert. Andererseits
ermöglicht uns diese Feststellung die Betrachtung der Kategorie $\Adj$, wobei
$\Ob(\Adj)\defeq\Ob(\Cat)$ und $\Adj(\catA,\catB)$ definiert ist als die 
Menge aller Adjunktionen % TODO: check
zwischen den kleinen Kategorien $\catA$ und $\catB$;
die Verknüpfung von Morphismen ist die Komposition von Adjunktionen, die wir 
gleich besprechen werden, und die naheliegenden Adjunktion des
Identitätsfunktors zu sich selbst ist der Identitäsmorphimus auf einer kleinen
Kategorie. Außerdem gibt es eine zusätzliche Struktur auf $\Adj$, die diese
Kategorie zu einer sogenannten $2$-Kategorie macht. Dies geht jedoch über den
Inhalt dieses Skripts hinaus, so dass wir an dieser Stelle auf 
Mac~Lane\cite[\S\,IV.8,\;\S\,XII.3]{bookc:maclane97} und 
nLab\cite{www:nlab:2category} verweisen.

\begin{thProposition}[Komposition von Adjunktionen]
    \label{ch2:kompos}
    Seien $\catA,\catB$ und $\catC$ Kategorien und seien $F,G,\bar F,\bar G$
    Funktoren wie in \eqref{ch2:situation}.
    Dann gilt auch $\bar F F \leftadjoint G\bar G$.
    %
    Genauer gilt:
    \begin{itemize}
        \item
            Sind $(F,G,\phi)$ und $(\bar F,\bar G,\bar\phi)$ zugehörige
            Adjunktionen zu obiger Situation, so ist auch
            \[ (\bar F F, G\bar G, \tilde\phi) \]  
            eine Adjunktion, wobei $\tilde\phi$ für alle 
            $A\in\catA,\; C\in\catC$ gegeben ist durch
            \[ \tilde\phi_{A,C} \defeq 
                \phi_{A,\bar GC} \after \bar\phi_{F\!A,C}
            . \]

        \item
            Sind $(F,G,\eta,\epsilon)$ und $(\bar F, \bar G, \bar\eta,
            \bar\epsilon)$ zugehörige Adjunktionen zu obiger Situation, so ist
            auch
            \[ (\bar F F, G\bar G, \tilde\eta, \tilde\epsilon) \]
            eine Adjunktion, wobei $\tilde\eta$ und $\tilde\epsilon$ gegeben
            sind durch
            \[ \tilde\eta \defeq G\bar\eta F \after \eta
                \qqundqq
                \tilde\epsilon \defeq \bar\epsilon \after \bar F\epsilon \bar G
            . \]
    \end{itemize}
\end{thProposition}

\begin{proof}
    Seien $(F,G,\phi)$ und $(\bar F,\bar G,\bar\phi)$ wie in der Behauptung.
    Dann haben wir zunächst für alle $A\in\catA,\; C\in\catC$ Isomorphismen
    \[  \catC(\bar F F\!A, C) \cong 
        \catB(F\!A,\bar G C)  \cong 
        \catA(A, G\bar G C)
    , \]
    was direkt aus den gegebenen Adjunktionen folgt. Dass dies natürlich in $A$
    und $C$ passiert, ist anhand unserer Definition in \ref{ch1:def:adjunktion}
    auch leicht ersichtlich, denn wir brauchen nur die natürlichen
    Transformationen
    \[ \bar\phi\,(F\op\times\Id_\catC)\colon
        \catC(\bar F F\blank,\blank) \nattrafoto \catB(F\blank,\bar G\blank)
    \]
    und
    \[ \phi\,(\Id_\catA\op\times\bar G)\colon  
        \catB(F\blank,\bar G\blank) \nattrafoto \catA(\blank,G\bar G\blank)
    \]
    zu verknüpfen und erhalten so eine natürliche Transformation~$\tilde\phi$,
    die auf Objekten gerade der Verknüpfung von $\phi$ und $\bar\phi$ wie in der
    Behauptung entspricht.

    \smallskip\noindent
    Seien nun $(F,G,\eta,\epsilon)$ und $(\bar F, \bar G, \bar\eta,
    \bar\epsilon)$ sowie $\tilde\eta,\,\tilde\epsilon$ wie in der Behauptung.
    Wir müssen zeigen, dass die natürlichen Transformationen
    $\tilde\eta,\tilde\epsilon$ die Dreiecksidentitäten erfüllen.
    Wir zeigen nur
    \[ \tilde\epsilon\bar FF \after \bar FF\tilde\eta = \id_{\bar FF}  \mkern2mu; \]
    die andere Gleichung rechnet man mit analogen Argumenten nach.
    Sei also $A\in\catA$. Dann ist zu zeigen, dass
    \[ \tilde\epsilon_{\bar FF\!A} \after \bar FF(\tilde\eta_A) = \id_{\bar FF\!A} \]
    gilt. Wir rechnen:
    \begin{align*}
        \tilde\epsilon_{\bar FF\!A} \after \bar FF(\tilde\eta_A)
        %
        &\overset{\scriptscriptstyle(1)}=
            \bigl( \bar\epsilon_{\bar FF\!A} \after 
                \bar F(\epsilon_{\bar G\bar FF\!A}) \bigr)
            \after
            \bar FF\bigl( G(\bar\eta_{F\!A})\after \eta_A \bigr)
        \\
        &\overset{\scriptscriptstyle(2)}= 
            \bar\epsilon_{\bar FF\!A} \after 
            \bar F\bigl( 
            \epsilon_{\bar G\bar FF\!A} \after
            FG(\bar\eta_{F\!A})\after F(\eta_A) \bigr)
        \\
        &\overset{\scriptscriptstyle(3)}= 
            \bar\epsilon_{\bar FF\!A} \after 
            \bar F\bigl( 
            \bar\eta_{F\!A}\after \epsilon_{F\!A}
            \after F(\eta_A) \bigr)
        \\
        &\overset{\scriptscriptstyle(4)}= 
            \bar\epsilon_{\bar FF\!A} \after 
            \bar F( \bar\eta_{F\!A}\after \id_{F\!A} )
        \\
        &\overset{\scriptscriptstyle(5)}= 
            \bar\epsilon_{\bar FF\!A} \after 
            \bar F(\bar\eta_{F\!A})
        \\
        &\overset{\scriptscriptstyle(6)}= 
            \id_{\bar FF\!A}
    \end{align*}
    Dabei wurden die Voraussetzungen wie folgt benutzt:
    \begin{enumerate}[(1)]
        \item
            Definition von $\tilde\epsilon$ und $\tilde\eta$
        \item
            Funktorialität von $\bar F$ und $F$
        \item
            $\epsilon\colon FG\nattrafoto\Id_\catB$ natürliche Transformation
        \item
            Dreiecksidentität der Adjunktion $(F,G,\eta,\epsilon)$
        \item
            Neutralität von $\id_{F\!A}$
        \item
            Dreiecksidentität der Adjunktion 
            $(\bar F,\bar G,\bar\eta,\bar\epsilon)$
    \end{enumerate}
\end{proof}

\begin{thBeispiel}[Nullobjekte]
    \newcommand{\cO}{\circledchar[black!40]{$0$}}
    %
    Sei $\catA$ eine punktierte Kategorie, d.\,h. eine Kategorie, die ein
    Nullobjekt\footnote{zugleich initiales und terminales Objekt}~$0\in\catA$ 
    besitzt. Dann wissen wir nach \cref{ch1:bsp:initterm},\footnote{für die
    verwendete Terminologie, siehe dort} dass der Funktor $C\colon\catA\to\One$ 
    sowohl rechts- als auch linksadjungiert zu $\cO\colon\One\to\catA$ ist:

    \[ 
        \xymatrix{ \catA \ar@<3.5pt>[r]^C & \ar@<2pt>[l]^{\cO} \One
        \ar@<3.5pt>[r]^{\cO} & \ar@<2pt>[l]^C \catA
        }
        \qquad\text{mit}\quad 
        C\leftadjoint\cO \text{\; und \,} \cO\leftadjoint C
    . \]

    \noindent
    Bilden wir nun die Komposition der Adjunktionen, so erhalten wir eine
    weitere Adjunktion $(\cO C, \cO C, \tilde\eta, \tilde\epsilon)$ des Funktors
    $\cO C\colon\catA\to\catA$ zu sich selbst. Einheit und Koeinheit der
    Adjunktionen $C\leftadjoint\cO$ und $\cO\leftadjoint C$ sind aus
    \cref{ch1:bsp:initterm} bekannt, also können wir $\tilde\eta$ und
    $\tilde\epsilon$ gemäß \cref{ch2:kompos} berechnen. Sei $A\in\catA$. 
    Dann gilt
    \begin{align*}
        \tilde\eta_A &= \cO(\id_{C\!A}) \after (A{\to}0)
        = \cO(\id_\ast) \after (A{\to}0) = \id_0 \after (A{\to}0) = (A{\to}0)
        \\
        \shortintertext{und}
        %
        \tilde\epsilon_A &= (0{\to}A) \after \cO(\id_{C\!A}) = (0{\to}A)
    . \end{align*}
    Also ist die Komponente der Einheit~$\tilde\eta$ gerade der eindeutige
    Morphismus vom dem entsprechenden Objekt zum Nullobjekt und die Komponente
    der Koeinheit~$\tilde\epsilon$ ist der eindeutige Morphismus vom Nullobjekte
    zum entsprechenden Objekt.
\end{thBeispiel}
