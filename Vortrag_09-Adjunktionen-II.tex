\documentclass[11pt,a4paper,ngerman,DIV=11,bibliography=totoc,titlepage=false]{scrreprt}

%%%%%%%%%%%%%%%%%%%%%%%%%%%%%%%%%%%%%%%%%%%%%%%%%%%%%%%%%%%%%%%%%%%%%
%%% packages
%%%%%%%%%%%%%%%%%%%%%%%%%%%%%%%%%%%%%%%%%%%%%%%%%%%%%%%%%%%%%%%%%%%%%

\usepackage[utf8]{inputenc}
\usepackage[T1]{fontenc}
\usepackage[ngerman]{babel}

\usepackage{amsmath}
\usepackage{amssymb}
\usepackage{amsthm}
\usepackage{mathtools}
\usepackage[all]{xy}
\usepackage{tikz}

\usepackage[babel]{csquotes}
\usepackage[shortlabels]{enumitem}
\usepackage[numbers,sort&compress]{natbib}
\usepackage{ifmtarg}
\usepackage{xstring}
\usepackage{remreset}


\usepackage[pdftex,bookmarks,colorlinks=false,pdfborder={0 0 0},%
            pdftitle={Seminar zu Kagegorien - %
                      Vortrag 9: Adjunktionen II},%
            pdfauthor={Johannes Prem}]{hyperref}
%
\usepackage{cleveref}
\let\cref=\Cref

\usepackage{myhelpers}  % my own myhelpers.sty
\usepackage{mymathmisc} % my own mymathmisc.sty


%%%%%%%%%%%%%%%%%%%%%%%%%%%%%%%%%%%%%%%%%%%%%%%%%%%%%%%%%%%%%%%%%%%%%
%%% macro definitions and other things
%%%%%%%%%%%%%%%%%%%%%%%%%%%%%%%%%%%%%%%%%%%%%%%%%%%%%%%%%%%%%%%%%%%%%

% don't reset footnote numbers
% (uses the 'remreset' package)
\makeatletter
\@removefromreset{footnote}{chapter}
\makeatother


% make parenthesized versions of \ref and cleveref's \cref
\newcommand*{\pref}[1]{(\ref{#1})}
\newcommand*{\pcref}[1]{(\cref{#1})}
\newcommand*{\pmycref}[1]{(\mycref{#1})}

% make a even more clever \mycref that produces "Lemma 42(a)" etc.
\newcommand{\mycref}[1]{%
    \begingroup%
    \StrCount{#1}{:}[\mycrefCount]%
    \StrBefore[\mycrefCount]{#1}{:}[\myrefMain]%
    \expandafter\cref\expandafter{\myrefMain}\,\ref{#1}%
    \endgroup%
}

% make \varepsilon, \varphi and \varrho default
\varifygreekletters{\epsilon\phi\rho}

% change the qedsymbol to my favoured blacksquare
\renewcommand{\qedsymbol}{$\blacksquare$}

% style for /all/ theorem like environments
\newtheoremstyle{mythms}
 {15pt}% space above
 {12pt}% space below 
 {}% body font
 {}% indent amount
 {\bfseries}% theorem head font
 {.}% punctuation after theorem head
 {0.6cm plus 0.25cm minus 0.1cm}% space after theorem head (\newline possible)
 {}% theorem head spec 
 
% set style and define thm like environments
\theoremstyle{mythms}
\newtheorem{globalnum}{DUMMY DUMMY DUMMY}[chapter]
\newtheorem{thDef}[globalnum]{Definition}
%\newtheorem{thNotation}[globalnum]{Notation}
\newtheorem{thSatz}[globalnum]{Satz}
\newtheorem{thProposition}[globalnum]{Proposition}
\newtheorem{thLemma}[globalnum]{Lemma}
\newtheorem{thKorollar}[globalnum]{Korollar}

\newtheorem{thErinnerung}[globalnum]{Erinnerung}
\newtheorem{thErinnerDef}[globalnum]{Erinnerung/Definition}
\newtheorem{thBemerkung}[globalnum]{Bemerkung}
%\newtheorem{thWarnung}[globalnum]{Warnung}
\newtheorem{thBeispiel}[globalnum]{Beispiel}
\newtheorem{thBeispiele}[globalnum]{Beispiele}
\newenvironment{BspList}[2][]{%
\nopagebreak\begin{thBeispiele}#1%
\hfill\begin{enumerate}[#2,parsep=0pt,itemsep=0.8ex,leftmargin=2em]%
}{%
\end{enumerate}\end{thBeispiele}
}
%

% also define a 'proofsketch' version of 'proof'
\newenvironment{proofsketch}[1][]{%
\begin{proof}[Beweisskizze#1]
}{%
\end{proof}
}

% inject pdfbookmarks at thm like environments
\makeatletter
\let\origthmhead=\thmhead
\renewcommand{\thmhead}[3]{%
\origthmhead{#1}{#2}{#3}%
\belowpdfbookmark{#1\@ifnotempty{#1}{ }#2\thmnote{ (#3)}}{#1#2}%
}
\makeatother

% new math operators
\DeclareMathOperator*{\bigdotcup}{\overset{\mkern0mu\scalebox{0.6}{$\bullet$}}{\bigcup}}

% new math 'operators'
\newcommand{\sDMO}[1]{\expandafter\DeclareMathOperator\csname#1\endcsname{#1}}

\sDMO{colim}
\sDMO{id}
\sDMO{Id}
\sDMO{Ob}
\DeclareMathOperator{\powerset}{\mathcal{P}}
\DeclareMathOperator{\Topo}{\mathcal{T}}

% categories
\newcommand{\MakeCategoryName}[1]{%
    \expandafter\DeclareMathOperator\csname#1\endcsname{\mathsf{#1}}
}
\newcommand{\MakeCategoryPseudoName}[1]{%
    \expandafter\DeclareMathOperator\csname CAT#1\endcsname{\mathsf{#1}}
}

\MakeCategoryName{Ab}
\MakeCategoryName{Adj}
\MakeCategoryName{Cat}
\MakeCategoryName{Group}
\MakeCategoryName{Met}
\MakeCategoryName{CMet}
\MakeCategoryName{Mod}
\MakeCategoryName{Ring}
\MakeCategoryName{Set}
\MakeCategoryName{Top}
\MakeCategoryName{Vect}
\MakeCategoryPseudoName{Ord}
\MakeCategoryPseudoName{Subgroup}

\DeclareMathOperator{\One}{\mathsf{1}}

\newcommand{\Ord}[1]{\CATOrd_{#1}}
\newcommand{\Subgroup}[1]{\CATSubgroup_{#1}}

%
\newcommand{\lXX}[2]{\mathop{{}_{#2}\mkern-2.5mu#1}}
\newcommand{\makeLRcat}[1]{%
    \expandafter\newcommand\csname l#1\endcsname{\expandafter\lXX\csname#1\endcsname}
    \expandafter\newcommand\csname r#1\endcsname[1]{\csname#1\endcsname_{##1}}
}
%
\makeLRcat{Mod}

% symbols used for categories
\newcommand{\cat}{\mathcal}
%
\newcommand{\makecatshortcut}[1]{%
    \expandafter\newcommand\csname cat#1\endcsname{\cat{#1}}
}
\makeatletter
\@callforeachtoken\makecatshortcut{A B C D}
\makeatother

% make quantors that use \limits per default
\DeclareMathOperator*{\Exists}{\exists}
\DeclareMathOperator*{\forAll}{\forall}

% define an 'abs', 'norm' and 'Spann' command
\DeclarePairedDelimiter{\abs}{\lvert}{\rvert}
\DeclarePairedDelimiter{\norm}{\lVert}{\rVert}
\DeclarePairedDelimiter{\Spann}{\langle}{\rangle}

% define missing arrows
\newcommand{\longto}{\longrightarrow}
\newcommand{\longhookrightarrow}{\lhook\joinrel\relbar\joinrel\rightarrow}
\newcommand{\isorightarrow}[1][]{\xrightarrow[#1]{\smash{\raisebox{-2pt}{$\sim$}}}}
\newcommand{\mapsfrom}{\mathrel{\reflectbox{$\mapsto$}}}
\newcommand{\longmapsfrom}{\mathrel{\reflectbox{$\longmapsto$}}}

% provide mathbb symbols \N \Z \Q \R and \C
\defmathbbsymbols{N Z Q C}
\defmathbbsymbolsubs{R}

% quotient by means of groups/rings/vector spaces
\newcommand{\Quot}[3][\big]{%
\raisebox{2pt}{$\mathsurround=0pt\displaystyle #2$}\mkern-3mu%
#1/%
\mkern-3mu\raisebox{-3.5pt}{$\displaystyle #3$}%
}
\newcommand{\QuotS}[3][]{%
\raisebox{2pt}{$\mathsurround=0pt\displaystyle #2$}\mkern-1mu%
#1/%
\mkern-3mu\raisebox{-3.5pt}{$\displaystyle #3$}%
}

\newcommand{\ZQuot}[2][\big]{\Quot[#1]{\Z}{#2\Z}}
\newcommand{\txtZQuot}[1]{\Z/#1\Z}

% define some point set topology specific macros
\newcommand{\setclosure}[1]{\overline{#1}}
\newcommand{\setinterior}[1]{#1^\circ}
\newcommand{\setboundary}[1]{\partial #1}

% just some shortcuts and aliases
\newcommand{\after}{\surround{\mskip4mu plus 2mu minus 1mu}{\mathord{\circ}}}
\newcommand{\blank}{{-}}
\newcommand{\defeq}{\coloneqq}
\newcommand{\eqdef}{\eqqcolon}
\newcommand{\half}{\frac{1}{2}}
\newcommand{\leftadjoint}{\dashv}
\newcommand{\mr}{\mathrm}
\newcommand{\mf}{\mathfrak}
\newcommand{\nattrafoto}{\mathrel{\Rightarrow}}
\newcommand{\op}{^\mathsf{op}}
\newcommand{\pot}[1]{\powerset(#1)}
\newcommand{\setOneto}[1]{\{1,\ldots,#1\}}
\newcommand{\setZeroto}[1]{\{0,\ldots,#1\}}
\newcommand{\surround}[2]{#1#2#1}
\newcommand{\thalf}{\tfrac{1}{2}}

% some text shortcuts
% (uses 'myhelpers')
\qXq{iff}
\qXq{implies}
\qTXq{oder}
\qTXq{und}
\qqTXqq{und}

%
\newcommand{\Achtung}{\emph{Achtung:} }

% xy tip selection (ComputerModern)
\SelectTips{cm}{}
\UseTips

% xy specific settings
\newcommand{\xyhookdirspacing}{4pt}
\newdir{`}{\dir^{(}} 
\newdir{ `}{{}*!/-\xyhookdirspacing/\dir{`}}
\iffalse)\fi % fix syntax highlighting

% make circled chars with tikz
\newcommand{\circledchar}[2][black]{%
\tikz[
    baseline={(B.south)},
    mysep/.style={
           inner xsep=0.35pt,
           inner ysep=0.8pt
    }
]{
    \node[draw=#1,shape=circle,mysep]
        (N) at (0,0) {\tiny #2};
    \node [mysep] (B) at (N) {\tiny\phantom{#2}};
}}

% listing with -- is nicer than with bullets 
\setlist[itemize,1]{label=--}

% start at chapter 0
\setcounter{chapter}{-1}

%%%%%%%%%%%%%%%%%%%%%%%%%%%%%%%%%%%%%%%%%%%%%%%%%%%%%%%%%%%%%%%%%%%%%
%%% document
%%%%%%%%%%%%%%%%%%%%%%%%%%%%%%%%%%%%%%%%%%%%%%%%%%%%%%%%%%%%%%%%%%%%%

\begin{document}


\subject{Seminar: Kategorien}
\title{Adjunktionen II}
\author{Johannes Prem}
\date{11.12.2013}

\maketitle
\thispagestyle{empty}

\vfill
\begin{center}\footnotesize%
    (Der \LaTeX-Quellcode für dieses Skript befindet sich auf github:
    \url{https://github.com/J0J0/talk_about_adjunctions}\,)
\end{center}
\newpage


\chapter{Vorwort und Notation}
\ldots

\bigskip
Wir treffen folgende Vereinbarung:
Alle Kategorien seien \emph{lokal klein}, d.\,h. die Morphismen zwischen
zwei Objekten einer Kategorie bilden stets eine Menge (also ein Objekt
in $\Set$).

\bigskip
In diesem Skript wird folgende Notation verwendet:
\begin{itemize}
    \item
        Bekannte Kategorien (wobei wir nur die Objekte notieren):
        
        \hspace{6mm}
        \begin{tabular}{l@{\qquad}l}
            \emph{Bezeichnung} & \emph{Objekte}                         \\[2pt]
            $\Set$      &   Mengen                                      \\
            $\Top$      &   topologische Räume                          \\
            $\rMod R$   &   Moduln über einem kommutativen Ring~$R$      \\
            $\Cat$      &   kleine Kategorien
        \end{tabular}

    \item
        Weitere auftrende Kategorien werden mit $\cat A,\cat B, \cat C, \dots$
        bezeichnet. Sei $\catA$ eine Kategorie. Wir schreiben $A\in\catA$
        für ein Objekt~$A$ in $\catA$ und $\Ob(\catA)$ für die Klasse aller
        Objekte in $\catA$. Zu zwei Objekten $A,A'\in\catA$ bezeichne
        $\catA(A,A')$ die Menge aller Morphismen von $A$ nach $A'$.
        Anstatt von $f\in\catA(A,A')$ schreiben wir auch $f\colon A\to A'$ für
        einen Morphismus von $A$ nach $A'$.

    \item
        Sowohl $\subset$ als auch $\subseteq$ stehen für: enthalten oder gleich.
        Echt enthalten wird durch $\subsetneq$ gekennzeichnet.
    
    \item
        Die \emph{natürlichen Zahlen $\N$} beginnen mit $0$.
\end{itemize}

\chapter{Erinnerung und Beispiele}
Aus dem letzten Vortrag (siehe Loher\cite{talk:loher}) kennen wir das Konzept
\emph{adjungierter Funktoren}. Der Vollständigkeit halber besprechen wir noch
einmal knapp die verschiedenen Definitionsmöglichkeiten für Adjunktionen, welche
wir im Folgenden benutzen wollen.

\begin{thErinnerung}[Hom-Funktor(en)]
    Sei $\catA$ eine Kategorie. Dann ist $\catA(\blank,\blank)$ ein Funktor
    $\catA\op\times\catA\to\Set$, gegeben auf Objekten durch die offensichtliche
    Art und Weise und auf Morphismen wie folgt: seien $(A,B),(A',B')$ Objekte
    in $\catA\op\times\catA$ und sei $(f\op,g)\colon (A,B)\to (A',B')$ ein 
    Morphismus zwischen diesen, dann gilt:
    \[ \catA(f\op,g)\colon \catA(A,B)\to \catA(A',B'), \quad 
        h\mapsto g\after h\after f
    . \]
    Fixieren wir ein Objekt~$A\in\catA$, so ist
    also $\catA(A,\blank)$ ein kovarianter Funktor $\catA\to\Set$
    und $\catA(\blank,A)$ ein kontravarianter Funktor $\catA\to\Set$.
\end{thErinnerung}

\begin{thErinnerDef}[Adjungierte Funktoren, Adjunktion]
    \label{ch1:def:adjunktion}
    %
    Seien $F\colon \cat A\to\cat B$ und $G\colon \cat B\to \cat A$
    Funktoren zwischen Kategorien $\cat A, \cat B$. Wir nennen
    $F$ \emph{linksadjungiert zu $G$} und $G$ \emph{rechtsadjungiert zu $F$},
    wenn eine beiden folgenden äquivalenten Bedingungen erfüllt ist:

    \begin{description}
        \item[Morphismenmengen-Adjunktion]\hfill\\
            Es existiert ein natürlicher Isomorphismus 
            \[ \phi\colon \catB(F\blank,\blank)
                \nattrafoto \catA(\blank,G\blank)
            \]
            zwischen den Funktoren
            \begin{align*}
                \catB(\blank,\blank)\after(F\op\times\Id_\catB)&\colon
                \catA\op\times\catB\to\Set 
                \qquad\text{und}\\
                \catA(\blank,\blank)\after(\Id_\catA\op\times G)&\colon
                \catA\op\times\catB\to\Set
            . \end{align*}
            Wir beschreiben diese Situation auch wie folgt: Für alle Objekte
            $A\in\catA,\,B\in\catB$ ist
            \[ \phi_{A,B}\colon \catB(FA,B)\isorightarrow\catA(A,GB) \] 
            eine Bijektion und diese ist \emph{natürlich in $A$ und $B$}.

        \item[Einheit-Koeinheit-Adjunktion]\hfill\\
            Es existieren natürliche Transformationen
            \[  \eta\colon \Id_\catA \nattrafoto GF  \qundq
                \epsilon\colon FG \nattrafoto \Id_\catB 
             \]
            mit $\epsilon F \after F\eta = \id_F$ und 
            $G\epsilon \after \eta G = \id_G$ (\emph{Dreiecksidentitäten}, siehe
            \cref{ch1:fig:dreiecksid}). Wir nennen dann $\eta$ die
            \emph{Einheit} und $\epsilon$ die \emph{Koeinheit} der Adjunktion.
            %
            \begin{figure}
                \begin{equation*}
                    \begin{gathered} % treats vertical alignment
                        \xymatrix{
                            F \ar[r]^-{F\eta} \ar[dr]_-{\id_F} 
                            & FGF \ar[d]^(.43){\epsilon F}
                            \\ & F
                        }
                    \end{gathered}
                    \hspace{1.2cm}
                    \begin{gathered}
                        \xymatrix{
                            G \ar[r]^-{\eta G} \ar[dr]_-{\id_G} 
                            & GFG \ar[d]^(.43){G\epsilon}
                            \\ & G
                        }
                    \end{gathered}
                \end{equation*}
                \caption{Dreiecksidentitäten für Einheit und Koeinheit}
                \label{ch1:fig:dreiecksid}
            \end{figure}
    \end{description}
    
    \noindent
    Wir nennen
    \begin{itemize}
        \item 
            im ersteren Fall ein Tripel $(F,G,\phi)$,
        \item
            im zweiteren Fall ein Tupel $(F,G,\eta,\epsilon)$
    \end{itemize}
    eine \emph{Adjunktion}. In beiden Fällen schreiben wir $F\leftadjoint G$.
\end{thErinnerDef}

Einen Beweis für die Äquivalenz der Bedingungen findet man bei
Loher\cite{talk:loher} oder in einem beliebigen Buch zur Kategorientheorie.
Auch einige Beispiele wie  
\enquote{Frei $\leftadjoint$ Vergiss}, \enquote{Produkt $\leftadjoint$ Exponential}
oder \enquote{Tensor $\leftadjoint$ Hom} haben wir schon bei 
Loher\cite[1.3,\;1.4,\;2.8]{talk:loher} gesehen. Wir betrachten zunächst weitere
Beispiele, bevor wir allgemeine Eigenschaften von Adjunktionen näher
untersuchen.

\begin{thBeispiel}[Vergissfunktor auf \texorpdfstring{$\Top$}{Top}]
    Sei $U\colon\Top\to\Set$ der Vergissfunktor, der einem topologischen Raum
    seine unterliegende Menge zuordnet. Seien weiter $D,K\colon\Set\to\Top$ die
    Funktoren, die eine Menge mit der diskreten bzw. der
    Klumpentopologie\footnote{auch indiskrete oder chaotische Topologie genannt}
    ausstatten und so zu einem topologischen Raum machen. Alle drei Funktoren
    bilden Morphismen auf sich selbst\footnote{streng genommen: auf dieselbe
    Abbildung als Morphismus in der jeweils anderen Kategorie} ab.
    Sei $(Y,T)$ ein topologischer Raum und seien $(X,T_D)$ bzw. $(Z,T_K)$
    topologische Räume mit der diskreten bzw. der Klumpentopologie. Dann gilt
    \[  \Top\bigl( (X,T_D), (Y,T) \bigr) = \Set(X,Y)  \qundq
        \Top\bigl( (Y,T), (Z,T_D) \bigr) = \Set(X,Y)
    , \]
    denn jede Abbildung aus einem Raum mit der diskreten Topologie und jede
    Abbildung in einen Raum mit der Klumpentopologie ist stetig. Dies zeigt,
    dass die Funktoren $D,K$ auf Morphismen wohldefiniert sind. Wir behaupten
    nun, dass wir eine Kette von Adjunktionen $D\leftadjoint U\leftadjoint K$
    haben. Dies lässt leicht überprüfen, denn es gilt für alle Mengen~$Y$ und
    alle topologischen Räume $(X,T)$:
    \begin{alignat*}{2}
        \Top\bigl( DY, (X,T) \bigr) &= \Set(Y,X) &&= \Set\bigl( Y, U(X,T)\bigr)
        \quad\text{und}\\
        \Set\bigl( U(X,T), Y \bigr) &= \Set(X,Y) &&= \Top\bigl( (X,T), KY \bigr)
    . \end{alignat*}
    Also liefern $\phi^D_{Y,(X,T)} \defeq \id_{\Set(Y,X)}$ und 
    $\phi^K_{(X,T),Y} \defeq \id_{\Set(X,Y)}$ Adjunktionen $(D,U,\phi^D)$
    und $(U,K,\phi^K)$ (-- die Natürlichkeit ist klar).
\end{thBeispiel}

\pagebreak[2]
\begin{thBeispiel}[Initiale und terminale Objekte]
    \label{ch1:bsp:initterm}
    Sei $\catA$ eine Kategorie und sei $\ast$ das einzige Objekt der
    Kategorie~$\One$.\footnote{Kategorie mit genau einem Objekt und genau einem
    Morphismus} Sei weiter $C$ der eindeutig bestimmte Funktor $\catA\to\One$.
    Dann besitzt $\catA$ genau dann ein terminales Objekt, wenn es einen
    zu $C$ rechtsadjungierten Funktor gibt, und $\catA$ besitzt genau dann ein
    initiales Objekt, wenn es einen zu $C$ linksadjungierten Funktor gibt.

    \newcommand{\cA}{\circledchar[black!40]{$A$}}
    \newcommand{\cX}{\circledchar[black!40]{$X$}}
    \newcommand{\cY}{\circledchar[black!40]{$Y$}}
    Wir bemerken zunächst, dass ein Funktor $\One\to\catA$ nichts weiter ist
    als ein Objekt in $\catA$. Daher schreiben wir im Folgenden einfach
    $\cA$ für den Funktor $\One\to\catA$, der $\ast$ auf das Objekt~$A$ aus
    $\catA$ abbildet.
    
    Sei $\cX\colon\One\to\catA$ ein Funktor mit $C\leftadjoint\cX$. Ist dann
    $A\in\catA$ ein Objekt aus $\catA$, so gilt:
    \[ \One(\ast,\ast) = \One(C\!A,\ast) \cong \catA(A,\cX\ast) = \catA(A,X) . \]
    Da $\One(\ast,\ast) = \{\id_\ast\}$ genau aus dem Identitätsmorphismus auf
    $\ast$ besteht, muss es also genau ein Element in $\catA(A,X)$ geben, d.\,h.
    es gibt genau einen Morphismus $A\to X$. Da $A\in\catA$ beliebig war, ist
    $X$ also ein terminales Objekt. Analog erhalten wir für einen Funktor
    $\cY\colon\One\to\catA$ mit $\cY\leftadjoint C$ für alle $A\in\catA$:
    \[ \catA(Y,A) = \catA(\cY\ast,A) \cong \One(\ast,C\!A) = \One(\ast,\ast) . \]
    Damit ist $Y$ ein initiales Objekt in $\catA$.

    Sei nun umgekehrt $X$ ein terminales Objekt in $\catA$. Dann gilt
    $C\leftadjoint\cX$ vermöge der Adjunktion
    $\bigl(C,\cX, \eta, \epsilon\bigr)$ mit 
    \[ \eta \defeq (A{\to}X)_{A\in\catA} \qqundqq \epsilon \defeq (\id_\ast)_\ast
    . \]
    Die Natürlichkeit von $\epsilon$ ist klar und die von $\eta$ prüfen wir
    schnell nach: Sei $f\colon A\to A'$ ein Morphismus in $\catA$. Dann müssen
    wir zeigen, dass das Diagramm 
    \vspace{-2mm}
    \[
        \xymatrix@=3.5mm{A \ar[r]^f \ar[d] & A' \ar[d] \\ X \ar[r] & X}
    \]
    kommutiert. Aber da $X$ ein terminales Objekt ist, bleibt diesem Diagramm
    gar nichts anderes übrig, als zu kommutieren. Wir rechnen außerdem nach,
    dass $\eta$ und $\epsilon$ die Dreiecksidentitäten erfüllen. 
    Sei $A\in\catA$. Dann gilt
    \[ (\epsilon C \after C\eta)_A 
        = \epsilon_\ast \after C(A{\to}X)
        = \id_\ast \after \id_\ast
        = \id_\ast = \id_{C\!A}
    \]
    sowie
    \[ (\cX\mkern1mu\epsilon \after \eta\mkern1mu\cX)_\ast
        = \cX(\id_\ast) \after \eta_X
        = \id_X \after (X{\to}X)
        = \id_X = \id_{\cX\ast}
    . \]
    Völlig analog zeigt man, dass für ein initiales Objekt $Y\in\catA$ das Tupel
    \[ \bigl( \cY,\; C,\; (\id_\ast)_\ast,\; (Y{\to}A)_{A\in\catA} \bigr) \]
    eine Adjunktion $\cY\leftadjoint C$ definiert.
\end{thBeispiel}

\chapter{Komposition von Adjunktionen}
Wir wollen nun untersuchen, wie wir Adjunktionen verknüpfen können. Einerseits
ist es genrell praktisch zu wissen, dass für Kategorien $\catA,\catB,\catC$ und
Funktoren
\vspace{-1mm}
\[ \tag{$\star$} \label{ch2:situation}
    \xymatrix{ \catA \ar@<3.5pt>[r]^F & \ar@<2pt>[l]^{G\vphantom{\bar G}} \catB 
    \ar@<3.5pt>[r]^{\bar F} & \ar@<2pt>[l]^{\bar G} \catC
    }
    \qquad\text{mit}\quad 
    F\leftadjoint G \text{\; und \,} \bar F\leftadjoint\bar G
\]
auch eine Adjunktion der komponierten Funktoren existiert. Andererseits
ermöglicht uns diese Feststellung die Betrachtung der Kategorie $\Adj$, wobei
$\Ob(\Adj)\defeq\Ob(\Cat)$ und $\Adj(\catA,\catB)$ definiert ist als die 
Menge aller Adjunktionen % TODO: check
zwischen den kleinen Kategorien $\catA$ und $\catB$;
die Verknüpfung von Morphismen ist die Komposition von Adjunktionen, die wir 
gleich besprechen werden, und die naheliegenden Adjunktion des
Identitätsfunktors zu sich selbst ist der Identitäsmorphimus auf einer kleinen
Kategorie. Außerdem gibt es eine zusätzliche Struktur auf $\Adj$, die diese
Kategorie zu einer sogenannten $2$-Kategorie macht. Dies geht jedoch über den
Inhalt dieses Skripts hinaus, so dass wir an dieser Stelle auf 
Mac~Lane\cite[\S\,IV.8,\;\S\,XII.3]{bookc:maclane97} und 
nLab\cite{www:nlab:2category} verweisen.

\begin{thProposition}[Komposition von Adjunktionen]
    \label{ch2:kompos}
    Seien $\catA,\catB$ und $\catC$ Kategorien und seien $F,G,\bar F,\bar G$
    Funktoren wie in \eqref{ch2:situation}.
    Dann gilt auch $\bar F F \leftadjoint G\bar G$.
    %
    Genauer gilt:
    \begin{itemize}
        \item
            Sind $(F,G,\phi)$ und $(\bar F,\bar G,\bar\phi)$ zugehörige
            Adjunktionen zu obiger Situation, so ist auch
            \[ (\bar F F, G\bar G, \tilde\phi) \]  
            eine Adjunktion, wobei $\tilde\phi$ für alle 
            $A\in\catA,\; C\in\catC$ gegeben ist durch
            \[ \tilde\phi_{A,C} \defeq 
                \phi_{A,\bar GC} \after \bar\phi_{F\!A,C}
            . \]

        \item
            Sind $(F,G,\eta,\epsilon)$ und $(\bar F, \bar G, \bar\eta,
            \bar\epsilon)$ zugehörige Adjunktionen zu obiger Situation, so ist
            auch
            \[ (\bar F F, G\bar G, \tilde\eta, \tilde\epsilon) \]
            eine Adjunktion, wobei $\tilde\eta$ und $\tilde\epsilon$ gegeben
            sind durch
            \[ \tilde\eta \defeq G\bar\eta F \after \eta
                \qqundqq
                \tilde\epsilon \defeq \bar\epsilon \after \bar F\epsilon \bar G
            . \]
    \end{itemize}
\end{thProposition}

\begin{proof}
    Seien $(F,G,\phi)$ und $(\bar F,\bar G,\bar\phi)$ wie in der Behauptung.
    Dann haben wir zunächst für alle $A\in\catA,\; C\in\catC$ Isomorphismen
    \[  \catC(\bar F F\!A, C) \cong 
        \catB(F\!A,\bar G C)  \cong 
        \catA(A, G\bar G C)
    , \]
    was direkt aus den gegebenen Adjunktionen folgt. Dass dies natürlich in $A$
    und $C$ passiert, ist anhand unserer Definition in \ref{ch1:def:adjunktion}
    auch leicht ersichtlich, denn wir brauchen nur die natürlichen
    Transformationen
    \[ \bar\phi\,(F\op\times\Id_\catC)\colon
        \catC(\bar F F\blank,\blank) \nattrafoto \catB(F\blank,\bar G\blank)
    \]
    und
    \[ \phi\,(\Id_\catA\op\times\bar G)\colon  
        \catB(F\blank,\bar G\blank) \nattrafoto \catA(\blank,G\bar G\blank)
    \]
    zu verknüpfen und erhalten so eine natürliche Transformation~$\tilde\phi$,
    die auf Objekten gerade der Verknüpfung von $\phi$ und $\bar\phi$ wie in der
    Behauptung entspricht.

    \smallskip\noindent
    Seien nun $(F,G,\eta,\epsilon)$ und $(\bar F, \bar G, \bar\eta,
    \bar\epsilon)$ sowie $\tilde\eta,\,\tilde\epsilon$ wie in der Behauptung.
    Wir müssen zeigen, dass die natürlichen Transformationen
    $\tilde\eta,\tilde\epsilon$ die Dreiecksidentitäten erfüllen.
    Wir zeigen nur
    \[ \tilde\epsilon\bar FF \after \bar FF\tilde\eta = \id_{\bar FF}  \mkern2mu; \]
    die andere Gleichung rechnet man mit analogen Argumenten nach.
    Sei also $A\in\catA$. Dann ist zu zeigen, dass
    \[ \tilde\epsilon_{\bar FF\!A} \after \bar FF(\tilde\eta_A) = \id_{\bar FF\!A} \]
    gilt. Wir rechnen:
    \begin{align*}
        \tilde\epsilon_{\bar FF\!A} \after \bar FF(\tilde\eta_A)
        %
        &\overset{\scriptscriptstyle(1)}=
            \bigl( \bar\epsilon_{\bar FF\!A} \after 
                \bar F(\epsilon_{\bar G\bar FF\!A}) \bigr)
            \after
            \bar FF\bigl( G(\bar\eta_{F\!A})\after \eta_A \bigr)
        \\
        &\overset{\scriptscriptstyle(2)}= 
            \bar\epsilon_{\bar FF\!A} \after 
            \bar F\bigl( 
            \epsilon_{\bar G\bar FF\!A} \after
            FG(\bar\eta_{F\!A})\after F(\eta_A) \bigr)
        \\
        &\overset{\scriptscriptstyle(3)}= 
            \bar\epsilon_{\bar FF\!A} \after 
            \bar F\bigl( 
            \bar\eta_{F\!A}\after \epsilon_{F\!A}
            \after F(\eta_A) \bigr)
        \\
        &\overset{\scriptscriptstyle(4)}= 
            \bar\epsilon_{\bar FF\!A} \after 
            \bar F( \bar\eta_{F\!A}\after \id_{F\!A} )
        \\
        &\overset{\scriptscriptstyle(5)}= 
            \bar\epsilon_{\bar FF\!A} \after 
            \bar F(\bar\eta_{F\!A})
        \\
        &\overset{\scriptscriptstyle(6)}= 
            \id_{\bar FF\!A}
    \end{align*}
    Dabei wurden die Voraussetzungen wie folgt benutzt:
    \begin{enumerate}[(1)]
        \item
            Definition von $\tilde\epsilon$ und $\tilde\eta$
        \item
            Funktorialität von $\bar F$ und $F$
        \item
            $\epsilon\colon FG\nattrafoto\Id_\catB$ natürliche Transformation
        \item
            Dreiecksidentität der Adjunktion $(F,G,\eta,\epsilon)$
        \item
            Neutralität von $\id_{F\!A}$
        \item
            Dreiecksidentität der Adjunktion 
            $(\bar F,\bar G,\bar\eta,\bar\epsilon)$
    \end{enumerate}
\end{proof}

\begin{thBeispiel}[Nullobjekte]
    \newcommand{\cO}{\circledchar[black!40]{$0$}}
    %
    Sei $\catA$ eine punktierte Kategorie, d.\,h. eine Kategorie, die ein
    Nullobjekt\footnote{zugleich initiales und terminales Objekt}~$0\in\catA$ 
    besitzt. Dann wissen wir nach \cref{ch1:bsp:initterm},\footnote{für die
    verwendete Terminologie, siehe dort} dass der Funktor $C\colon\catA\to\One$ 
    sowohl rechts- als auch linksadjungiert zu $\cO\colon\One\to\catA$ ist:

    \[ 
        \xymatrix{ \catA \ar@<3.5pt>[r]^C & \ar@<2pt>[l]^{\cO} \One
        \ar@<3.5pt>[r]^{\cO} & \ar@<2pt>[l]^C \catA
        }
        \qquad\text{mit}\quad 
        C\leftadjoint\cO \text{\; und \,} \cO\leftadjoint C
    . \]

    \noindent
    Bilden wir nun die Komposition der Adjunktionen, so erhalten wir eine
    weitere Adjunktion $(\cO C, \cO C, \tilde\eta, \tilde\epsilon)$ des Funktors
    $\cO C\colon\catA\to\catA$ zu sich selbst. Einheit und Koeinheit der
    Adjunktionen $C\leftadjoint\cO$ und $\cO\leftadjoint C$ sind aus
    \cref{ch1:bsp:initterm} bekannt, also können wir $\tilde\eta$ und
    $\tilde\epsilon$ gemäß \cref{ch2:kompos} berechnen. Sei $A\in\catA$. 
    Dann gilt
    \begin{align*}
        \tilde\eta_A &= \cO(\id_{C\!A}) \after (A{\to}0)
        = \cO(\id_\ast) \after (A{\to}0) = \id_0 \after (A{\to}0) = (A{\to}0)
        \\
        \shortintertext{und}
        %
        \tilde\epsilon_A &= (0{\to}A) \after \cO(\id_{C\!A}) = (0{\to}A)
    . \end{align*}
    Also ist die Komponente der Einheit~$\tilde\eta$ gerade der eindeutige
    Morphismus vom dem entsprechenden Objekt zum Nullobjekt und die Komponente
    der Koeinheit~$\tilde\epsilon$ ist der eindeutige Morphismus vom Nullobjekte
    zum entsprechenden Objekt.
\end{thBeispiel}

\chapter{Adjungierte Funktoren erhalten Limiten}
Wir kommen nun zu einem der wichtigsten Aspekte von Adjunktionen: adjungierte
Funktoren erhalten (Ko)Limiten. Genauer:

\begin{thSatz}[Adjungierte Funktoren erhalten (Ko)Limiten]
    \label{ch3:rapl}
    %
    Seien $\catA,\catB$ Kategorien und $F\colon\catA\to\catB,\;
    G\colon\catB\to\catA$ Funktoren mit $F\leftadjoint G$.
    Dann gilt:
    \begin{itemize}
        \item
            Der rechtsadjungierte Funktor~$G$ erhält Limiten.
        \item
            Der linksadjungierte Funktor~$F$ erhält Kolimiten.
    \end{itemize}
\end{thSatz}

Dies macht deutlich, wie nützlich es für uns ist, wenn wir wissen, dass ein
Funktor ein rechts- oder linksadjungierter Funktor ist. Insbesondere werden wir
dies im Folgenden auch an einigen Beispielen erkennen. Davon abgesehen liefert
\cref{ch3:rapl} aber auch \emph{das Kriterium} schlechthin, um zu zeigen, dass
ein Funktor keinen links- oder rechtsadjungierten Funktor besitzen kann:

\begin{thKorollar}[Nicht-Existenz adjungierter Funktoren]
    \label{ch3:raplkontra}
    %
    Sei $F\colon\catA\to\catB$ ein Funktor zwischen Kategorien $\catA,\catB$.
    Falls $F$ einen (bliebigen!) Limes (Kolimes) nicht erhält, 
    so besitzt $F$ keinen linksadjungierten (rechtsadjungierten) Funktor.
\end{thKorollar}

Das Korollar folgt sofort mit Kontraposition aus \cref{ch3:rapl}. Für den Beweis
des Satzes benötigen wir noch ein Lemma, zu dessen Formulierung wir wiederum den
Diagonalfunktor~$\Delta$ brauchen:

\begin{thErinnerDef}[Diagonalfunktor]
    \label{ch3:def:diagonalfunktor}
    %
    Sei $I$ eine kleine Kategorie und $\catA$ eine Kategorie.
    \begin{itemize}
        \item 
            Für alle $A\in\catA$ bezeichne $\Delta^I_A$ das $I$-Diagramm in
            $\catA$, welches alle Objekte aus $I$ auf $A$ abbildet und alle
            Morphismen auf $\id_A$.
            
        \item
            Der \emph{Diagonalfunktor $\Delta^I\colon\catA\to\catA^I$} ist 
            gegeben durch
            \[ A\mapsto \Delta^I_A \]
            auf Objekten und auf Morphismen wie folgt: Ist $f\colon A\to B$
            ein Morphismus von Objekten $A,B$ in $\catA$, so ist $\Delta^I(f)$
            die natürliche Transformation $\Delta^I_A \nattrafoto \Delta^I_B$,
            die auf jedem Objekt aus~$I$ durch $f$ gegeben ist.
            
        \item
            Ist $D\colon I\to\catA$ ein $I$-Diagramm in $\catA$ und $X\in\catA$,
            so ist ein Kegel $\bigl( X \to D(i) \bigr)_{i\in I}$ nichts anderes
            als eine natürliche Transformation $\Delta^I_X \nattrafoto D$.
            Insbesondere entspricht $\catA^I(\Delta^I_X, D)$ gerade der Menge aller
            Kegel über $D$.
    \end{itemize}
\end{thErinnerDef}

\begin{thLemma}
    \label{ch3:limesviadiagrammkategorie}
    %
    Sei $I$ eine kleine Kategorie, $\catA$ eine Kategorie und $D\colon
    I\to\catA$ ein $I$-Diagramm in $\catA$. Sei $X\in\catA$ und gelte
    \[ \catA(A,X) \cong \catA^I(\Delta^I A, D) \]
    natürlich in $A\in\catA$. Dann ist $X$ ein Limes von $D$ über $I$.
\end{thLemma}

\begin{proof}
    Sei $\phi$ ein expliziter natürlicher Isomorphismus
    \[ \catA^I(\Delta^I\blank, D) \nattrafoto \catA(\blank, X)  \]
    von Funktoren $\catA\op\to\Set$ (welcher nach Voraussetzung existiert).
    Sei weiter 
    \[ p \defeq \phi_X^{-1}(\id_X)\colon \Delta^I_X\nattrafoto D  . \]
    Wir behaupten nun, dass $(X,p)$ ein Limes von $D$ über $I$ ist. Sei also
    $Y\in\catA$ und $q\colon\Delta^I_Y\nattrafoto D$. Dann liefert 
    $f \defeq \phi_Y(q)$ einen Morphismus $Y\to X$ und wir müssen nur noch
    zeigen, dass für alle $i\in I$ schon $p_i\after f = q_i$ gilt
    (\cref{ch3:fig:trafos}, links) und dass es keine weiteren Mor\-phi\-smen
    $Y\to X$ mit dieser Eigenschaft gibt. Weil $\phi$ eine natürliche
    Transformation ist, gilt $f^*\after \phi_X = \phi_Y \after (\Delta^I f)^*$
    (d.\,h. das rechte Diagramm in \cref{ch3:fig:trafos} kommutiert).
    Nach Definition gilt $p = \phi_X^{-1}(\id_X)$ und damit
    \[ (f^*\after \phi_X)(p) = f^*(\id_X) = f  . \]
    Damit erhalten wir:
    \begin{align*}
        \phi_Y(q) = f
        &= \bigl( \phi_Y \after (\Delta^I f)^* \bigr)(p)    \\
        &= \phi_Y\bigl( p\after (\Delta^I f) \bigr)
    . \end{align*}
    Da $\phi_Y$ bijektiv ist, folgt $q = p \after (\Delta^I f)$ und dies
    bedeutet komponentenweise nichts anderes als 
    $q_i = \bigl( p \after (\Delta^I f) \bigr)_i = p_i \after f$
    für alle $i\in I$. Nehmen wir nun an, $g\colon Y\to X$ ist ein zweiter
    Morphismus mit $q = p\after (\Delta^I g)$. Dann ist $\tilde q\defeq
    \phi_Y^{-1}(g)$ eine natürliche Transformation, die analog zur Berechnung
    bei $q$ die Eigenschaft $\tilde q = p \after (\Delta^I g)$ erfüllt. Dann
    folgt aber schon $q = \tilde q$ und damit $f = \phi_Y(q) = 
    \phi_Y(\tilde q) = g$. Dies zeigt die Behauptung.
    \\
    %
    \begin{figure}[b]
        \centering
        \begin{equation*}
            \begin{gathered}
                \xymatrix{
                    & \ar@/_1pc/[ddl]_{q_i} Y \ar@/^1pc/[ddr]^{q_j} 
                    \ar[d]^f &
                    \\
                    & \ar[dl]_(.4){p_i} X \ar[dr]^(.4){p_j} &
                    \\
                    D(i) & \cdots & D(j)
                }
            \end{gathered}
            %
            \hspace{2cm}
            %
            \begin{gathered}
                \xymatrix{
                    \catA^I(\Delta^I_X, D) \ar[r]^{\phi_X} \ar[d]^{(\Delta^I f)^*} &
                    \catA(X,X) \ar[d]_{f^*}
                    \\
                    \catA^I(\Delta^I_Y, D) \ar[r]_{\phi_Y} & \catA(Y,X)
                }
            \end{gathered}
        \end{equation*}
        \caption{Situation im Beweis von \cref{ch3:limesviadiagrammkategorie}}
        \label{ch3:fig:trafos}
    \end{figure}
\end{proof}

\begin{proof}[Beweis von \cref{ch3:rapl}]
    \belowpdfbookmark{Beweis von Satz~\ref{ch3:rapl}}{beweisrapl}
    %
    Sei $I$ eine kleine Kategorie und $D\colon I\to\catB$ ein $I$-Diagramm
    in $\catB$, für das der Limes $\lim_I D$ existiert. 
    Wir geben nun zuerst einen einfachen Beweis dafür, dass $G$ diesen Limes
    erhält, falls $\catA$ eine vollständige Kategorie\footnote{d.\,h. es
    existieren alle kleinen Limiten (also Limiten von Diagrammen über kleinen
    Kategorien)} ist. Es gilt für alle Objekte $A\in\catA$:
    \begin{align*}
        \catA\bigl(A, G(\lim\nolimits_I D)\bigr)
        &\cong \catB(FA, \lim\nolimits_I D)          \\
        &\cong \lim\nolimits_I \catB(FA, D\blank)    \\
        &\cong \lim\nolimits_I \catA(A, GD\blank)    \\
        &\cong \catA(A,\lim\nolimits_I GD)
    \end{align*}
    Dabei haben wir zweimal die (natürliche) Isomorphie zwischen
    Morphismenmengen vermöge der Adjunktion $F\leftadjoint G$ ausgenutzt und
    zweimal, dass (kovariante) Hom-Funktoren mit Limiten vertauschen. Das
    Yoneda-Lemma liefert nun: 
    \[ G(\lim\nolimits_I D) \cong \lim\nolimits_I GD  . \]
    Den Fall für $F$ erhält man dual (unter Beachtung der Tatsache, dass der
    kontravariante Hom-Funktor Kolimiten auch zu Limiten macht), wenn man
    fordert, dass $\catB$ kovollständig ist.
    
    \noindent
    Nun zum allgemeinen Fall.  Dann ist der kritische Punkt in obigem Beweis,
    dass wir nicht wissen, ob \enquote{$\lim_I GD$} überhaupt
    existiert.\footnote{%
        Interessanterweise findet man in vielen Quellen trotzdem den obigen
        Beweis ohne jegliche Einschränkungen an $\catA$ oder $\catB$. Falls
        jemand eine (einfache) Rechtfertigung dafür kennt, möge sich diese
        Person bitte bei mir melden \ldots%
    }
    Wie im ersten Fall erhalten wir für alle $A\in\catA$
    \[  \catA\bigl(A, G(\lim\nolimits_I D)\bigr)
        \cong \lim\nolimits_I \catA(A, GD\blank)
    \]
    natürlich in $A$. Nun macht man sich leicht durch die explizite Darstellung
    von Limiten in $\Set$ klar, dass der Limes auf der rechten Seite gerade die
    Menge aller natürlichen Transformationen $\Delta^I_A \nattrafoto GD$ ist.
    Das heißt für alle $A\in\catA$ gilt
    \[ \catA\bigl(A, G(\lim\nolimits_I D)\bigr)
        \cong \catA^I(\Delta^I A, GD)
    , \]
    natürlich in $A$. Mit \cref{ch3:limesviadiagrammkategorie} erhalten wir also
    wie gewünscht:
    \[ G(\lim\nolimits_I D) \cong \lim\nolimits_I GD  . \]
\end{proof}

\begin{thBeispiel}[Vergissfunktoren vs. (Ko)Limiten]\hfill
    \begin{itemize}
        \item
            In \cref{ch1:bsp:TopVergiss} haben wir gesehen, dass der Vergissfunktor
            $U\colon\Top\to\Set$ sowohl einen links- als auch einen rechtsadjungierten
            Funktor besitzt. Nach \cref{ch3:rapl} erhält $U$ somit also alle(!) Limiten
            \emph{und} Kolimiten.
            
        \item
            Betrachten wir hingegen Vergissfunktoren von algebraischen Kategorien, so
            sehen wir, dass diese im Allgemeinen nicht ganz so gutartig sind. Nehmen wir
            beispielsweise den bei Loher\cite[1.3]{talk:loher} eingeführten
            Vergissfunktor $U\colon \Vect_k\to\Set$ von der Kategorie der Vektorräume
            über einem gegebenen Körper~$k$ nach $\Set$, so erhält dieser zwar alle
            Limiten (denn der freie Funktor $\Set\to\Vect_k$ ist ein zu $U$
            linksadjungierter Funktor), aber $U$ erhält nicht alle Kolimiten. Dies sehen
            wir zum Beispiel anhand des initialen Objekts $0\in\Vect_k$ (trivialer
            Vektorraum) ein, denn $U(0) = \{0\}$, aber eine einelementige Menge ist
            \emph{kein} initiales Objekt in $\Set$. Insbesondere wissen wir damit auch,
            dass $U$ tatsächlich überhaupt keinen rechtsadjungierten Funktor
            besitzen kann. \pcref{ch3:raplkontra}
    \end{itemize}
\end{thBeispiel}

\begin{thBeispiel}[Tensorprodukt vs. direkte Summe und Quotienten]
    \label{ch3:bsp:tensorvskolimiten}
    \newcommand{\tensorM}{\blank\otimes M}
    %
    Sei $R$ ein kommutativer Ring und $M\in\Mod_R$ ein $R$-Modul. Wenn wir im
    Folgenden $\otimes$ schreiben, so meinen wir stets das Tensorprodukt über
    $R$. 
    
    \noindent
    Wir betrachten den Funktor $\tensorM\colon\Mod_R\to\Mod_R$, von dem wir
    bereits wissen (siehe Loher\cite[2.8]{talk:loher}), dass er einen
    rechtsadjungierten Funktor besitzt (nämlich den Funktor
    $\operatorname{Hom}(M,\blank)\colon \Mod_R\to\Mod_R$). Also ist $\tensorM$
    ein linksadjungierter Funktor und als solcher erhält $\tensorM$ alle
    Kolimiten. Da die direkte Summe von $R$-Moduln gerade das Koprodukt (und
    damit ein Kolimes) in $\Mod_R$ ist, erhalten wir also sofort die äußerst
    nützliche Tatsache: Tensorieren vertauscht mit direkten Summen.  In Formeln
    bedeutet dies: Ist $(N_i)_{i\in J}$ eine Familie von $R$-Moduln, dann gilt
    \[ \Bigl(\mkern2mu \bigoplus_{i\in J} N_i \Bigr)\otimes M
        \cong \bigoplus_{i\in J} (N_i\otimes M)
    . \]
    Man kann dies natürlich auch konkret auf Elementen nachrechnen oder mehrfach
    die universellen Eigenschaften von $\oplus$ und $\otimes$ anwenden; jedoch
    ist der Beweis \enquote{$\tensorM$ ist ein linksadjungierter Funktor} doch
    deutlich kürzer und angenehmer.
    
    \medskip\noindent
    Sei nun $\mf a\triangleleft R$ ein Ideal in $R$. Wie man sich leicht klarmacht,
    ist der Pushout des Diagramms 
    \[ 
        \begin{gathered}
            \xymatrix@=4mm{ \mf a \ar[d] \ar@{ `->}[r] & R \\ 0 & } 
        \end{gathered}
        \qquad\rightsquigarrow\qquad
        \begin{gathered}
            \xymatrix@=4mm{ \mf a \ar[d] \ar@{ `->}[r] & R \ar[d] \\ 
            0 \ar[r] & R/\mf a }
        \end{gathered}
    \]
    gerade der Quotient $R/\mf a$ (wobei dies allgemein für Pushouts in $\Mod_R$
    gilt). Also können wir \enquote{Quotientenbildung} als Kolimes auffassen und
    nachdem $\tensorM$ ein linksadjungierter Funktor ist, muss also auch
    \[ 
        \begin{gathered}
            \xymatrix@=4mm{ \mf a\otimes M \ar[d] \ar[r] & R\otimes M \ar[d]
                           \\ 0\otimes M \ar[r] & (R/\mf a) \otimes M }
        \end{gathered}
        \qquad\rightsquigarrow\qquad
        \begin{gathered}
            \xymatrix@=4mm{ \mf a\otimes M \ar[d] \ar[r] & M \ar[d]
                           \\ 0 \ar[r] & (R/\mf a) \otimes M }
        \end{gathered}
    \]
    ein Pushout-Diagramm in $\Mod_R$ sein, mit
    \[ (R/\mf a) \otimes M \;\cong\; \Quot{R\otimes M}{\mf a\otimes M} . \]
    Also müssen wir nur wissen, wie $\mf a\otimes M$ als als Teilmenge von
    $R\otimes M \cong M$ aussieht. Man sieht leicht, dass unter dem
    Iasomorphismus (gegeben auf elementaren Tensoren durch)
    \[ R\otimes M \to M, \quad r\otimes m \mapsto rm \]
    das Bild von $\mf a\otimes M$ gerade $\mf aM\subset M$ entspricht.
    (Achtung, im Allgemeinen gilt jedoch \emph{nicht}
    $\mf a\otimes M \cong \mf aM$. Letztere Isomorphie ist aber
    insbesondere dann gegeben, wenn $M$ flach ist.) Also haben wir gezeigt:
    \[ (R/\mf a) \otimes M \;\cong\; \Quot{M}{\mf aM} \mkern1mu. \]
\end{thBeispiel}

\begin{thBeispiel}[Tensorprodukt vs. Limiten]
    \label{ch3:bsp:tensorvslimiten}
    \newcommand{\tensorQ}{\blank\otimes\Q}
    In \cref{ch3:bsp:tensorvskolimiten} haben wir gesehen, dass sich
    Tensorprodukte mit Kolimiten vertragen. Nun wollen wir zeigen, dass
    $\blank\otimes M$ (mit den Bezeichnern aus dem vorherigen Beispiel)
    Limiten und insbesondere Produkte im Allgemeinen nicht erhält, d.\,h.,
    dass \enquote{Tensorieren} nicht mit Limiten/Produkten vertauscht.

    \noindent
    Für diejenigen Leser, denen die \emph{$p$-adischen ganzen Zahlen~$\Z_p$} 
    für eine Primzahl $p\in\Z$ als projektiver Limes $\varprojlim_{n\in\N}
    \txtZQuot{p^n}$ bekannt sind, nennen wir gleich das interessante Beispiel:
    \[ \thickmuskip=8mu
        \Z_p\underset{\scriptscriptstyle\Z}\otimes\Q \cong \Q_p \not\cong 0 
        \cong  \varprojlim_{n\in\N} \,\Bigl( \ZQuot{p^n} 
        \underset{\scriptscriptstyle\Z}\otimes \Q \Bigr)
    . \]

    \noindent
    Wir wollen dieses Phänomen nun ohne Zuhilfenahme von $p$-adischen Zahlen
    etwas näher beleuchten.  Dazu betrachten wir den $\Z$-Modul $\Q$ und den
    Funktor $\tensorQ$ (wobei wir stets über $\Z$ tensorieren). Sei $p\in\Z$
    eine Primzahl. Weiter fassen wir $\N$ (zunächst) als diskrete
    Kategorie\footnote{d.\,h. als Katgorie, die außer den Identitätsmorphismen
    keine weiteren besitzt} auf und definieren das $\N$-Diagramm~$Z$ in
    $\Mod_\Z$ durch
    \[ \N\to\Mod_\Z, \quad n\mapsto \ZQuot{p^n}  \]
    auf Objekten (-- da es keine nicht-trivalen Morphismen gibt und Identitäten
    auf Identitäten abgebildet werden müssen gibt es auf Morphismen nichts zu
    tun). Dann ist der Limes von $Z$ über $\N$ in $\Mod_\Z$ gerade das folgende
    Produkt in $\Mod_\Z$:
    \[ \prod_{n\in\N} \ZQuot{p^n} \;=\; \lim_\N Z  . \]
    Für alle $n\in\N$ ist $\txtZQuot{p^n}$ offenbar ein Torsionsmodul über $\Z$,
    also
    \[ \Bigl( \ZQuot{p^n} \Bigr) \otimes \Q  \;\cong\; 0  . \]
    Insbesondere gilt somit:
    \[ \thickmuskip=8mu
        0 \cong \prod_{n\in\N} 0 
        \cong \prod_{n\in\N} \Bigl( \ZQuot{p^n} \otimes\Q\Bigr)
        = \lim_\N \,\bigl((Z\blank)\otimes\Q\bigr)
    . \]
    Aber es gilt
    \[ \Bigl( \lim_\N Z \Bigr) \otimes \Q 
        \;\cong\; \bigl(\Z\setminus\{0\}\bigr)^{-1} 
        \Bigl( \prod_{n\in\N} \ZQuot{p^n} \Bigr)
        \;\not\cong\; 0  , 
    \]
    denn $([1])_{n\in\N}/1$ ist ein nicht-trivales Element in der Lokalisierung
    $(\Z\setminus\{0\})^{-1} \bigl( \prod_{n\in\N} \txtZQuot{p^n} \bigr)$.
    Wenn wir nun noch klären, wie $\Z_p$ aus dem Eingangsbeispiel genau
    definiert ist, so lassen sich diese Argumente leicht darauf übertragen, was
    an dieser Stelle dem Leser zur Übung überlassen sei.
    
    \begin{thErinnerDef}[Ordnungskategorie, projektiver/induktiver Limes]%
        \label{ch3:def:ordkatprojindlim}
        \hfill\\
        Sei $(N,\leq)$ eine partiell geordnete Menge und sei $\catA$ eine
        Kategorie.
        \begin{itemize}
            \item
                Die (kleine) Kategorie $\Ord{(N,\leq)}$ sei gegeben durch
                $\Ob(\Ord{(N,\leq)}) \defeq N$ und für $i,j\in N$ gebe es
                einen eindeutigen Morphismus $i\to j$, falls $i\leq j$
                gilt (ansonsten keinen); sind $i,j,k\in N$ und $i\to j,\; j\to
                k$ Morphismen, so sei $(j\to k)\after (i\to j) \defeq (i\to k)$.
                
            \item
                Sei $(N,\leq)$ zusätzlich eine gerichtete Menge\footnote{d.\,h.
                eine partiell geordnete Menge und für alle $i,j\in N$ gebe es
                ein $k\in N$ mit $i\leq k$ und $j\leq k$} und sei $I \defeq
                \Ord{(N,\leq)}$.
                
                Sei $D\colon I\op\to\catA$ ein $I\op$-Diagramm in $\catA$. 
                Falls $D$ einen Limes über $I\op$ in $\catA$ besitzt, so
                definieren wir den \emph{projektiven Limes von $D$ über $N$}
                durch
                \[ \varprojlim_{n\in N} D(n) \defeq \lim\nolimits_{I\op} D  . \]
                
                Sei $\bar D\colon I\to\catA$ ein $I$-Diagramm in $\catA$. 
                Falls $\bar D$ einen Kolimes über $I$ in $\catA$ besitzt, so
                definieren wir den \emph{induktiven Limes von $\bar D$ über $N$}
                durch
                \[ \varinjlim_{n\in N} \bar D(n) 
                    \defeq \colim\nolimits_I \bar D  
                . \]
                %
                Wir nennen $D$ auch ein \emph{projektives System über $N$
                in $\catA$} und $\bar D$ ein \emph{induktives System über
                $N$ in $\catA$}.
        \end{itemize}
    \end{thErinnerDef}
    
    \begin{thDef}[Vervollständigung von Ringen, $p$-adische (ganze) Zahlen]\hfill
        \begin{itemize}
            \item
                Sei $R$ ein kommutativer Ring und $\mf a\triangleleft R$ ein
                Ideal. Dann definiert
                \begin{align*}
                    \N \ni n   &\mapsto \Quot{R}{\mf a^n}  \\[3pt]
                    (n\leq m)  &\mapsto \bigl(\mkern2mu
                        \Quot{R}{\mf a^m}  \to \Quot{R}{\mf a^n}, \;\;
                                       [x] \mapsto [x]
                    \mkern1mu \bigr)
                \end{align*}
                ein projektives System in $\Ring$ über $(\N,\leq)$ (mit der
                üblichen (Total-)Ordnung \enquote{$\leq$} auf~$\N$). Wir
                bezeichnen den projektiven Limes
                \[ \varprojlim_{n\in\N} \, \Bigl( \Quot{R}{\mf a^n} \Bigr) \]
                als \emph{$\mf a$-adische Vervollständigung von $R$}.
                
            \item
                Sei $p\in\Z$ eine Primzahl und $\Spann{p}$ das von $p$ in $\Z$
                erzeugte Ideal. Dann sind die \emph{$p$-adischen ganzen
                Zahlen~$\Z_p$} definiert als die $\Spann{p}$-adische
                Vervollständigung von $\Z$:
                \[ \Z_p \defeq \varprojlim_{n\in\N} \, \Bigl( \ZQuot{p^n} \Bigr)
                . \]
                Den Quotientenkörper $\operatorname{Quot}(\Z_p)$ des
                Integritätsrings~$\Z_p$ nennen wir \emph{$p$-adische Zahlen} und
                wir bezeichnen diesen mit $\Q_p$.
        \end{itemize}
    \end{thDef}
    
    \begin{thBemerkung}\hfill
        \begin{itemize}
            \item
                In $\Set,\Group,\Ring,\Mod_R,\Vect_k$ (für einen kommutativen
                Ring~$R$ und einen Körper~$k$) existieren stets alle projektiven
                Limiten und diese lassen sich wie folgt explizit angeben:
                Sei $(N,\leq)$ eine gerichtete Menge und 
                \[ \bigl( (X_i)_{i\in N},
                \, (f_{ij}\colon X_j\to X_i)_{i,j\in N,\, i\leq j} \bigr) \]
                ein
                projektives System über $(N,\leq)$ in einer der genannten
                Kategorien.\footnote{%
                    Diese Daten (zusammen mit gewissen Forderungen an die
                    Morphismen $f_{ij}$) bestimmen offenbar ein
                    $\Ord{(N,\leq)}\op$-Diagramm im Sinne von
                    \cref{ch3:def:ordkatprojindlim}.%
                } Dann gilt:
                \[ \varprojlim_{i\in N} X_i 
                    = \Bigl\{
                        (x_i)_{i\in N} \in \prod\nolimits_{i\in N} X_i
                        \cMid\Big \forall\,i,j\in N\colon\;
                        i \leq j \implies x_i = f_{ij}(x_j)
                    \Bigr\}
                \]
                (mit komponentenweisen Verknüpfungen).
                
            \item
                In \cref{ch3:bsp:tensorvslimiten} betrachten wir Limiten in
                $\Mod_\Z$, aber anhand der expliziten Darstellung wird klar,
                dass wir für $\varprojlim_{n\in\N} \txtZQuot{p^n}$ dieselbe
                unterliegende Menge in $\Ring$ und $\Mod_\Z$ erhalten (bzw.
                sogar dieselbe zugrunde liegende abelsche Gruppe) und diese 
                nur mit unterschiedlichen Verknüpfungen ausgestattet ist.
        \end{itemize}
    \end{thBemerkung}
\end{thBeispiel}

\begin{thErinnerung}[Limes als Funktor]
    Sei $\catA$ eine vollständige Kategorie und $I$ eine kleine Kategorie.
    Wir fixieren für alle $I$-Diagramme in $\catA$ einen fest gewählten Limes.
    Dann ist $\lim_I$ ein Funktor von der Diagrammkategorie~$\catA^I$ nach
    $\catA$. Sind $D$ und $\bar D$ zwei $I$-Diagramm in $\catA$ und
    $\eta\colon D\nattrafoto \bar D$ eine natürliche Transformation,
    so ist $\lim_I\eta$ der (eindeutige) Morphismus $\lim_I D\to\lim_I\bar D$,
    der das Diagramm in \cref{ch3:fig:limesalsfunktor} zum Kommutieren bringt.
    %
    \begin{figure}
        \centering
        \begin{equation*}
                \xymatrix{
                    &        \ar@/_1.5pc/[ddl]_{\eta_i\after p_i} 
                    \lim_I D \ar@/^1.5pc/[ddr]^{\eta_j\after p_j} 
                    \ar@{-->}[d]^(.57){\lim_I\eta} &
                    \\
                    & \ar[dl]_(.4){\bar p_i} \lim_I\bar D 
                      \ar[dr]^(.4){\bar p_j} &
                    \\
                    \bar D(i) \ar[rr]^{\bar D(h)}
                    \ar@{}@<-16pt>[rr]|-{i \xrightarrow[\text{in $I$}]{h} j}
                    & & \bar D(j)
                }
        \end{equation*}
        \caption{Limes als Funktor, wobei $\eta\colon D\nattrafoto\bar D$ eine
            natürliche Transformation ist und $(\bar p_i)_{i\in I}$ bzw.
            $(p_i)_{i\in I}$ die zu $\lim_I\bar D$ bzw. $\lim_I D$ gehörigen
            Morphismen sind.}
        \label{ch3:fig:limesalsfunktor}
    \end{figure}
\end{thErinnerung}

\begin{thBeispiel}[Limes als rechtsadjungierter Funktor]
    \label{ch3:bsp:limesrechtsadjungiert}
    %
    Sei $\catA$ eine vollständige Kategorie und $I$ eine kleine Kategorie.
    Wir fixieren für alle $I$-Diagramme in $\catA$ einen fest gewählten Limes.
    Dann ist $\lim_I$ rechtsadjungiert zum Diagonalfunktor~$\Delta^I$
    \pcref{ch3:def:diagonalfunktor}, d.\,h. es gilt
    \[ \Delta^I \leftadjoint \lim\nolimits_I  . \]
\end{thBeispiel}

\begin{proof}
    \let\origlim=\lim
    \renewcommand{\lim}{\origlim\nolimits}
    %
    \begin{figure}
        \centering
        \begin{equation*}
                \xymatrix{
                    & \ar@/_1.5pc/[ddl]_{\eta_i} 
                    A \ar@/^1.5pc/[ddr]^{\eta_j} 
                    \ar@{-->}[d]_(.57){\exists!}^(.57){g_D^A(\eta)} &
                    \\
                    & \ar[dl]_(.4){p_i} \lim_I D 
                      \ar[dr]^(.4){p_j} &
                    \\
                    D(i) \ar[rr]^{D(h)}
                    \ar@{}@<-16pt>[rr]|-{i \xrightarrow[\text{in $I$}]{h} j}
                    & & D(j)
                }
        \end{equation*}
        \caption{Zuordnung $\catA^I(\Delta^I A, D) \to \catA(A,\lim_I D)$ im
            Beweis von \cref{ch3:bsp:limesrechtsadjungiert}}
        \label{ch3:fig:nattrafozumorphismus}
    \end{figure}
    %
    Sei $A\in\catA$ und $D\in\catA^I$ mit Limes $\bigl(\lim_I D,(p_i)_{i\in I}\bigr)$.
    Wir überlegen uns zunächst, dass tatsächlich
    \[ \catA^I(\Delta^I A, D) \cong \catA(A,\lim_I D) \]
    gilt. Sei $\eta\in\catA^I(\Delta^I A, D)$, d.\,h. $\eta$ ist eine natürliche
    Transformation $\Delta^I_A\nattrafoto D$. Wir hatten uns schon überlegt,
    dass $(A,\eta)$ damit gerade ein Kegel über $D$ ist, so dass wir aus der
    universellen Eigenschaft des Limes einen eindeutigen Morphismus
    $g_D^A(\eta)\colon A\to \lim_I D$ bekommen (siehe
    \cref{ch3:fig:nattrafozumorphismus}). Ist umgekehrt $g\colon A\to \lim_I D$
    ein Morphismus in $\catA$, so definiert $(p_i\after g)_{i\in I}$ eine
    natürliche Transformation $\Delta^I_A \nattrafoto D$. Wie eben erhalten wir
    zu dieser einen eindeutigen Morphismus  $\tilde g\colon A\to\lim_I D$ und
    wegen der Eindeutigkeit in der universellen Eigenschaft des Limes muss dann
    schon $g=\tilde g$ gelten. Damit haben wir gezeigt, dass
    \[ \catA^I(\Delta^I A, D) \overset{g_D^A}\longto \catA(A,\lim_I D) \]
    eine Bijektion ist. Es bleibt also zu zeigen, dass dies auch natürlich in
    $A$ und $D$ passiert.\footnote{Man sieht leicht, dass es zu unserer
        Definition äquivalent ist, die Natürlichkeit in jeder Variable einzeln
        zu prüfen. (Siehe auch Youcis\cite{www:an:youcis:productcat2}.)%
    }
    Dies wollen wir im Folgenden einmal direkt nachrechnen.
    
    \begin{description}
        \item[Natürlichkeit in \boldmath$A$]\hfill\\
            Seien $B$ und $A$ Objekte in $\catA$, $f\in\catA(B,A)$ (Achtung,
            Kontravarianz) und $D\in\catA^I$ mit Limes $\bigl(\lim_I
            D,(p_i)_{i\in I}\bigr)$. Weil $D$ konstant ist, schreiben wir
            einfach $g^A \defeq g_D^A$ und $g^B \defeq g_D^B$.
            Wir haben zu zeigen, dass das linke Diagramm in
            \cref{ch3:fig:limrechtsadjdiagramme} kommutiert.  Sei dazu
            $\eta\in\catA^I(\Delta^I A, D)$. Dann gilt (als Morphismen
            $B\to\lim_I D$ in $\catA$)
            \begin{align*}
                \bigl(f^* \after g^A\bigr)(\eta)
                &= f^*\bigl( g^A(\eta) \bigr)
                 = g^A(\eta) \after f
                \qquad\text{und}\\[0.5ex]
                \bigl(g^B \after (\Delta^I f)^*\bigr)(\eta)
                &= g^B\bigl( \eta \after (\Delta^I f) \bigr)
            . \end{align*}
            Für alle $i\in I$ erhalten wir
            \begin{align*}
                p_i\after \bigl(g^A(\eta)\bigr)\after f 
                &= \eta_i\after f
                \qquad\text{und}\\[0.5ex]
                p_i\after \bigl(g^B\bigl( \eta \after (\Delta^I f) \bigr)\bigr)
                &= \bigl( \eta \after (\Delta^I f)\bigr)_i
                 = \eta_i \after f
            , \end{align*}
            weswegen nach der universellen Eigenschaft von $\lim_I D$
            schon 
            \[ \bigl(f^* \after g^A\bigr)(\eta) 
                = \bigl(g^B \after (\Delta^I f)^*\bigr)(\eta)
            \]
            gelten muss. Da $\eta$ beliebig war, folgt
            \[ f^* \after g^A = g^B \after (\Delta^I f)^* , \]
            und gerade das wollten wir zeigen.
            
        \item[Natürlichkeit in \boldmath$D$]\hfill\\
            Seien $D,\bar D\in\catA^I$, $\eta\in\catA^I(D,\bar D)$ und
            $A\in\catA$. Seien $\bigl(\lim_I D,(p_i)_{i\in I}\bigr)$ und 
            $\bigl(\lim_I \bar D,(\bar p_i)_{i\in I}\bigr)$ die Limiten von 
            $D$ bzw. $\bar D$ über $I$. Da hier $A$ konstant ist, schreiben wir
            nur $g_D \defeq g_D^A$ und $g_{\bar D} \defeq g_{\bar D}^A$.
            Nun haben wir zu zeigen, dass das rechte Diagramm in
            \cref{ch3:fig:limrechtsadjdiagramme} kommutiert.  Sei dazu
            $\alpha\in\catA^I(\Delta^I A, D)$. Dann gilt (als Morphismen
            $A\to\lim_I\bar D$ in $\catA$)
            \begin{align*}
                \bigl((\lim_I\eta)_\ast \after g_D \bigr)(\alpha)
                &= (\lim_I\eta)_\ast \bigl( g_D(\alpha) \bigr)
                 = (\lim_I\eta) \after g_D(\alpha)
                \qquad\text{und}\\[0.5ex]
                \bigl(g_{\bar D}\after\eta_\ast\bigr)(\alpha)
                &= g_{\bar D}\bigl( \eta_\ast(\alpha) \bigr)
                 = g_{\bar D}(\eta\after\alpha)
            . \end{align*}
            Für alle $i\in I$ erhalten wir
            \begin{align*}
                \bar p_i \after (\lim_I\eta) \after \bigl(g_D(\alpha)\bigr)
                &= \eta_i \after p_i \after \bigl(g_D(\alpha)\bigr)
                 = \eta_i \after \alpha_i
                \qquad\text{und}\\[0.5ex]
                \bar p_i \after \bigl(g_{\bar D}(\eta\after\alpha)\bigr)
                &= (\eta\after\alpha)_i = \eta_i\after\alpha_i
            , \end{align*}
            weswegen nach der universellen Eigenschaft von $\lim_I\bar D$
            schon
            \[ \bigl((\lim_I\eta)_\ast \after g_D \bigr)(\alpha)
                = \bigl(g_{\bar D}\after\eta_\ast\bigr)(\alpha)
            \]
            gelten muss. Da $\alpha$ beliebig war, folgt
            \[ (\lim_I\eta)_\ast \after g_D = g_{\bar D}\after\eta_\ast , \]
            und das wollten wir zeigen.
    \end{description}
    Also ist $g_D^A$ tatsächlich bijektiv und natürlich in $A$ und $D$ und damit
    ist alles gezeigt.
    \\
    %
    \begin{figure}
        \centering
        \begin{equation*}
            \begin{gathered}
                \xymatrix{
                    \catA^I(\Delta^I A, D) \ar[r]^{g^A} \ar[d]^{(\Delta^I f)^*} &
                    \catA(A, \lim_I D) \ar[d]_{f^*}
                    \\
                    \catA^I(\Delta^I B, D) \ar[r]_{g^B} & \catA(B, \lim_I D)
                }
            \end{gathered}
            %
            \hspace{2cm}
            %
            \begin{gathered}
                \xymatrix{
                    \catA^I(\Delta^I A, D) \ar[r]^{g_D} \ar[d]^{\eta_\ast} &
                    \catA(A, \lim_I D) \ar[d]_{(\lim_I\eta)_\ast}
                    \\
                    \catA^I(\Delta^I A, \bar D) \ar[r]_{g_{\bar D}} &
                    \catA(A, \lim_I\bar D)
                }
            \end{gathered}
        \end{equation*}
        \caption{Diagramme, deren Kommutativität im Beweis von
            \cref{ch3:bsp:limesrechtsadjungiert} zum Nachweis der Natürlichkeit
            gezeigt wird}
        \label{ch3:fig:limrechtsadjdiagramme}
    \end{figure}
\end{proof}

\begin{thBemerkung}\hfill
    \begin{itemize}
        \item
            Dual kann man zeigen, dass man \enquote{Kolimesbildung}
            als Funktor auffassen kann und dass dieser linksadjungiert
            zum Diagonalfunktor ist.
            
        \item
            Wir kennen nun einen zweiten Beweis dafür, dass in vollständigen
            Kategorien Limiten mit Limiten vertauschen, denn wir haben soeben
            gezeigt, dass man \enquote{Limesbildung} als rechtsadjungierten
            Funktor auffassen kann und als solcher vertauscht dieser nach
            \cref{ch3:rapl} mit Limiten.
    \end{itemize}
\end{thBemerkung}


\nocite{lecnotes:leinster}
\nocite{bookc:awodey10}
\nocite{bookc:maclane97}

\appendix
\bibliographystyle{plaindin}
\bibliography{bibsources}

\end{document}





