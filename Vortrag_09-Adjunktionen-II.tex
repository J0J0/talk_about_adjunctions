\documentclass[11pt,a4paper,ngerman,DIV=11,bibliography=totoc,titlepage=false]{scrreprt}

%%%%%%%%%%%%%%%%%%%%%%%%%%%%%%%%%%%%%%%%%%%%%%%%%%%%%%%%%%%%%%%%%%%%%
%%% packages
%%%%%%%%%%%%%%%%%%%%%%%%%%%%%%%%%%%%%%%%%%%%%%%%%%%%%%%%%%%%%%%%%%%%%

\usepackage[utf8]{inputenc}
\usepackage[T1]{fontenc}
\usepackage[ngerman]{babel}

\usepackage{amsmath}
\usepackage{amssymb}
\usepackage{amsthm}
\usepackage{mathtools}
\usepackage[all]{xy}
%\usepackage{tikz}

\usepackage[babel]{csquotes}
\usepackage[shortlabels]{enumitem}
\usepackage[numbers,sort&compress]{natbib}
\usepackage{ifmtarg}
\usepackage{xstring}
\usepackage{remreset}


\usepackage[pdftex,bookmarks,colorlinks=false,pdfborder={0 0 0},%
            pdftitle={Seminar zu Kagegorien - %
                      Vortrag 9: Adjunktionen II},%
            pdfauthor={Johannes Prem}]{hyperref}
%
\usepackage{cleveref}
\let\cref=\Cref

\usepackage{myhelpers}  % my own myhelpers.sty
\usepackage{mymathmisc} % my own mymathmisc.sty


%%%%%%%%%%%%%%%%%%%%%%%%%%%%%%%%%%%%%%%%%%%%%%%%%%%%%%%%%%%%%%%%%%%%%
%%% macro definitions and other things
%%%%%%%%%%%%%%%%%%%%%%%%%%%%%%%%%%%%%%%%%%%%%%%%%%%%%%%%%%%%%%%%%%%%%

% don't reset footnote numbers
% (uses the 'remreset' package)
\makeatletter
\@removefromreset{footnote}{chapter}
\makeatother


% make parenthesized versions of \ref and cleveref's \cref
\newcommand*{\pref}[1]{(\ref{#1})}
\newcommand*{\pcref}[1]{(\cref{#1})}
\newcommand*{\pmycref}[1]{(\mycref{#1})}

% make a even more clever \mycref that produces "Lemma 42(a)" etc.
\newcommand{\mycref}[1]{%
    \begingroup%
    \StrCount{#1}{:}[\mycrefCount]%
    \StrBefore[\mycrefCount]{#1}{:}[\myrefMain]%
    \expandafter\cref\expandafter{\myrefMain}\,\ref{#1}%
    \endgroup%
}

% make \varepsilon, \varphi and \varrho default
\varifygreekletters{\epsilon\phi\rho}

% change the qedsymbol to my favoured blacksquare
\renewcommand{\qedsymbol}{$\blacksquare$}

% style for /all/ theorem like environments
\newtheoremstyle{mythms}
 {15pt}% space above
 {12pt}% space below 
 {}% body font
 {}% indent amount
 {\bfseries}% theorem head font
 {.}% punctuation after theorem head
 {0.6cm plus 0.25cm minus 0.1cm}% space after theorem head (\newline possible)
 {}% theorem head spec 
 
% set style and define thm like environments
\theoremstyle{mythms}
\newtheorem{globalnum}{DUMMY DUMMY DUMMY}[chapter]
\newtheorem{thDef}[globalnum]{Definition}
%\newtheorem{thNotation}[globalnum]{Notation}
\newtheorem{thSatz}[globalnum]{Satz}
%\newtheorem{thPropos}[globalnum]{Proposition}
\newtheorem{thLemma}[globalnum]{Lemma}
\newtheorem{thKorollar}[globalnum]{Korollar}

\newtheorem{thErinnerung}[globalnum]{Erinnerung}
\newtheorem{thErinnerDef}[globalnum]{Erinnerung/Definition}
\newtheorem{thBemerkung}[globalnum]{Bemerkung}
%\newtheorem{thWarnung}[globalnum]{Warnung}
\newtheorem{thBeispiel}[globalnum]{Beispiel}
\newtheorem{thBeispiele}[globalnum]{Beispiele}
\newenvironment{BspList}[2][]{%
\nopagebreak\begin{thBeispiele}#1%
\hfill\begin{enumerate}[#2,parsep=0pt,itemsep=0.8ex,leftmargin=2em]%
}{%
\end{enumerate}\end{thBeispiele}
}
%

% also define a 'proofsketch' version of 'proof'
\newenvironment{proofsketch}[1][]{%
\begin{proof}[Beweisskizze#1]
}{%
\end{proof}
}

% inject pdfbookmarks at thm like environments
\makeatletter
\let\origthmhead=\thmhead
\renewcommand{\thmhead}[3]{%
\origthmhead{#1}{#2}{#3}%
\belowpdfbookmark{#1\@ifnotempty{#1}{ }#2\thmnote{ (#3)}}{#1#2}%
}
\makeatother

% new math operators
\DeclareMathOperator*{\bigdotcup}{\overset{\mkern0mu\scalebox{0.6}{$\bullet$}}{\bigcup}}

% new math 'operators'
\newcommand{\sDMO}[1]{\expandafter\DeclareMathOperator\csname#1\endcsname{#1}}

\sDMO{id}
\sDMO{Id}
\sDMO{Ob}
\DeclareMathOperator{\powerset}{\mathcal{P}}
\DeclareMathOperator{\Topo}{\mathcal{T}}

% categories
\newcommand{\MakeCategoryName}[1]{%
    \expandafter\DeclareMathOperator\csname#1\endcsname{\mathsf{#1}}
}
\newcommand{\MakeCategoryPseudoName}[1]{%
    \expandafter\DeclareMathOperator\csname CAT#1\endcsname{\mathsf{#1}}
}

\MakeCategoryName{Ab}
\MakeCategoryName{Cat}
\MakeCategoryName{Group}
\MakeCategoryName{Met}
\MakeCategoryName{CMet}
\MakeCategoryName{Mod}
\MakeCategoryName{Set}
\MakeCategoryName{Top}
\MakeCategoryName{Vect}
\MakeCategoryPseudoName{Ord}
\MakeCategoryPseudoName{Subgroup}

\newcommand{\Ord}[1]{\CATOrd_{#1}}
\newcommand{\Subgroup}[1]{\CATSubgroup_{#1}}

%
\newcommand{\lXX}[2]{\mathop{{}_{#2}\mkern-2.5mu#1}}
\newcommand{\makeLRcat}[1]{%
    \expandafter\newcommand\csname l#1\endcsname{\expandafter\lXX\csname#1\endcsname}
    \expandafter\newcommand\csname r#1\endcsname[1]{\csname#1\endcsname_{##1}}
}
%
\makeLRcat{Mod}

% symbols used for categories
\newcommand{\cat}{\mathcal}
%
\newcommand{\makecatshortcut}[1]{%
    \expandafter\newcommand\csname cat#1\endcsname{\cat{#1}}
}
\makeatletter
\@callforeachtoken\makecatshortcut{A B C D}
\makeatother

% make quantors that use \limits per default
\DeclareMathOperator*{\Exists}{\exists}
\DeclareMathOperator*{\forAll}{\forall}

% define an 'abs', 'norm' and 'Spann' command
\DeclarePairedDelimiter{\abs}{\lvert}{\rvert}
\DeclarePairedDelimiter{\norm}{\lVert}{\rVert}
\DeclarePairedDelimiter{\Spann}{\langle}{\rangle}

% define missing arrows
\newcommand{\longto}{\longrightarrow}
\newcommand{\longhookrightarrow}{\lhook\joinrel\relbar\joinrel\rightarrow}
\newcommand{\isorightarrow}[1][]{\xrightarrow[#1]{\smash{\raisebox{-2pt}{$\sim$}}}}
\newcommand{\mapsfrom}{\mathrel{\reflectbox{$\mapsto$}}}
\newcommand{\longmapsfrom}{\mathrel{\reflectbox{$\longmapsto$}}}

% provide mathbb symbols \N \Z \Q \R and \C
\defmathbbsymbols{Z Q C}
\defmathbbsymbolsubs{N R}

% define some point set topology specific macros
\newcommand{\setclosure}[1]{\overline{#1}}
\newcommand{\setinterior}[1]{#1^\circ}
\newcommand{\setboundary}[1]{\partial #1}

% just some shortcuts and aliases
\newcommand{\after}{\surround{\mskip4mu plus 2mu minus 1mu}{\mathord{\circ}}}
\newcommand{\blank}{{-}}
\newcommand{\defeq}{\coloneqq}
\newcommand{\eqdef}{\eqqcolon}
\newcommand{\half}{\frac{1}{2}}
\newcommand{\leftadjoint}{\dashv}
\newcommand{\mr}{\mathrm}
\newcommand{\nattrafoto}{\mathrel{\Rightarrow}}
\newcommand{\op}{^\mathsf{op}}
\newcommand{\pot}[1]{\powerset(#1)}
\newcommand{\setOneto}[1]{\{1,\ldots,#1\}}
\newcommand{\setZeroto}[1]{\{0,\ldots,#1\}}
\newcommand{\surround}[2]{#1#2#1}
\newcommand{\thalf}{\tfrac{1}{2}}

% some text shortcuts
% (uses 'myhelpers')
\qXq{iff}
\qXq{implies}
\qTXq{oder}
\qTXq{und}
\qqTXqq{und}

%
\newcommand{\Achtung}{\emph{Achtung:} }

% xy tip selection (ComputerModern)
\SelectTips{cm}{}
\UseTips

% listing with -- is nicer than with bullets 
\setlist[itemize,1]{label=--}

% start at chapter 0
\setcounter{chapter}{-1}

%%%%%%%%%%%%%%%%%%%%%%%%%%%%%%%%%%%%%%%%%%%%%%%%%%%%%%%%%%%%%%%%%%%%%
%%% document
%%%%%%%%%%%%%%%%%%%%%%%%%%%%%%%%%%%%%%%%%%%%%%%%%%%%%%%%%%%%%%%%%%%%%

\begin{document}


\subject{Seminar: Kategorien}
\title{Adjunktionen II}
\author{Johannes Prem}
\date{11.12.2013}

\maketitle
\thispagestyle{empty}

\vfill
\begin{center}\footnotesize%
    (Der \LaTeX-Quellcode für dieses Skript befindet sich auf github:
    \url{https://github.com/J0J0/talk_about_adjunctions}\,)
\end{center}
\newpage


\chapter{Vorwort und Notation}
\ldots

\bigskip
Wir treffen folgende Vereinbarung:
Alle Kategorien seien \emph{lokal klein}, d.\,h. die Morphismen zwischen
zwei Objekten einer Kategorie bilden stets eine Menge (also ein Objekt
in $\Set$).

\bigskip
In diesem Skript wird folgende Notation verwendet:
\begin{itemize}
    \item
        Bekannte Kategorien (wobei wir nur die Objekte notieren):
        
        \hspace{6mm}
        \begin{tabular}{l@{\qquad}l}
            \emph{Bezeichnung} & \emph{Objekte}                         \\[2pt]
            $\Set$      &   Mengen                                      \\
            $\Top$      &   topologische Räume                          \\
            $\rMod R$   &   Moduln über einem kommutativen Ring~$R$      \\
            $\Cat$      &   kleine Kategorien
        \end{tabular}

    \item
        Weitere auftrende Kategorien werden mit $\cat A,\cat B, \cat C, \dots$
        bezeichnet. Sei $\catA$ eine Kategorie. Wir schreiben $A\in\catA$
        für ein Objekt~$A$ in $\catA$ und $\Ob(\catA)$ für die Klasse aller
        Objekte in $\catA$. Zu zwei Objekten $A,A'\in\catA$ bezeichne
        $\catA(A,A')$ die Menge aller Morphismen von $A$ nach $A'$.
        Anstatt von $f\in\catA(A,A')$ schreiben wir auch $f\colon A\to A'$ für
        einen Morphismus von $A$ nach $A'$.

    \item
        Sowohl $\subset$ als auch $\subseteq$ stehen für: enthalten oder gleich.
        Echt enthalten wird durch $\subsetneq$ gekennzeichnet.
    
    \item
        Die \emph{natürlichen Zahlen $\N$} beginnen mit $0$.
\end{itemize}

\chapter{Erinnerung und Beispiele}
Aus dem letzten Vortrag (siehe Loher\cite{talk:loher}) kennen wir das Konzept
\emph{adjungierter Funktoren}. Der Vollständigkeit halber besprechen wir noch
einmal knapp die verschiedenen Definitionsmöglichkeiten für Adjunktionen, welche
wir im Folgenden benutzen wollen.

\begin{thErinnerung}[Hom-Funktor(en)]
    Sei $\catA$ eine Kategorie. Dann ist $\catA(\blank,\blank)$ ein Funktor
    $\catA\op\times\catA\to\Set$, gegeben auf Objekten durch die offensichtliche
    Art und Weise und auf Morphismen wie folgt: seien $(A,B),(A',B')$ Objekte
    in $\catA\op\times\catA$ und sei $(f\op,g)\colon (A,B)\to (A',B')$ ein 
    Morphismus zwischen diesen, dann gilt:
    \[ \catA(f\op,g)\colon \catA(A,B)\to \catA(A',B'), \quad 
        h\mapsto g\after h\after f
    . \]
    Fixieren wir ein Objekt~$A\in\catA$, so ist
    also $\catA(A,\blank)$ ein kovarianter Funktor $\catA\to\Set$
    und $\catA(\blank,A)$ ein kontravarianter Funktor $\catA\to\Set$.
\end{thErinnerung}

\begin{thErinnerDef}[Adjungierte Funktoren, Adjunktion]
    \label{ch1:def:adjunktion}
    %
    Seien $F\colon \cat A\to\cat B$ und $G\colon \cat B\to \cat A$
    Funktoren zwischen Kategorien $\cat A, \cat B$. Wir nennen
    $F$ \emph{linksadjungiert zu $G$} und $G$ \emph{rechtsadjungiert zu $F$},
    wenn eine beiden folgenden äquivalenten Bedingungen erfüllt ist:

    \begin{description}
        \item[Morphismenmengen-Adjunktion]\hfill\\
            Es existiert ein natürlicher Isomorphismus 
            \[ \phi\colon \catB(F\blank,\blank)
                \nattrafoto \catA(\blank,G\blank)
            \]
            zwischen den Funktoren
            \begin{align*}
                \catB(\blank,\blank)\after(F\op\times\Id_\catB)&\colon
                \catA\op\times\catB\to\Set 
                \qquad\text{und}\\
                \catA(\blank,\blank)\after(\Id_\catA\op\times G)&\colon
                \catA\op\times\catB\to\Set
            . \end{align*}
            Wir beschreiben diese Situation auch wie folgt: Für alle Objekte
            $A\in\catA,\,B\in\catB$ ist
            \[ \phi_{A,B}\colon \catB(FA,B)\isorightarrow\catA(A,GB) \] 
            eine Bijektion und diese ist \emph{natürlich in $A$ und $B$}.

        \item[Einheit-Koeinheit-Adjunktion]\hfill\\
            Es existieren natürliche Transformationen
            \[  \eta\colon \Id_\catA \nattrafoto GF  \qundq
                \epsilon\colon FG \nattrafoto \Id_\catB 
             \]
            mit $\epsilon F \after F\eta = \id_F$ und 
            $G\epsilon \after \eta G = \id_G$ (\emph{Dreiecksidentitäten}, siehe
            \cref{ch1:fig:dreiecksid}). Wir nennen dann $\eta$ die
            \emph{Einheit} und $\epsilon$ die \emph{Koeinheit} der Adjunktion.
            %
            \begin{figure}
                \begin{equation*}
                    \begin{gathered} % treats vertical alignment
                        \xymatrix{
                            F \ar[r]^-{F\eta} \ar[dr]_-{\id_F} 
                            & FGF \ar[d]^(.43){\epsilon F}
                            \\ & F
                        }
                    \end{gathered}
                    \hspace{1.2cm}
                    \begin{gathered}
                        \xymatrix{
                            G \ar[r]^-{\eta G} \ar[dr]_-{\id_G} 
                            & GFG \ar[d]^(.43){G\epsilon}
                            \\ & G
                        }
                    \end{gathered}
                \end{equation*}
                \caption{Dreiecksidentitäten für Einheit und Koeinheit}
                \label{ch1:fig:dreiecksid}
            \end{figure}
    \end{description}
    
    \noindent
    Wir nennen
    \begin{itemize}
        \item 
            im ersteren Fall ein Tripel $(F,G,\phi)$,
        \item
            im zweiteren Fall ein Tupel $(F,G,\eta,\epsilon)$
    \end{itemize}
    eine \emph{Adjunktion}. In beiden Fällen schreiben wir $F\leftadjoint G$.
\end{thErinnerDef}

Einen Beweis für die Äquivalenz der Bedingungen findet man bei
Loher\cite{talk:loher} oder in einem beliebigen Buch zur Kategorientheorie.
Auch einige Beispiele wie  
\enquote{Frei $\leftadjoint$ Vergiss}, \enquote{Produkt $\leftadjoint$ Exponential}
oder \enquote{Tensor $\leftadjoint$ Hom} haben wir schon bei 
Loher\cite[1.3,\;1.4,\;2.8]{talk:loher} gesehen. Wir betrachten zunächst weitere
Beispiele, bevor wir allgemeine Eigenschaften von Adjunktionen näher
untersuchen.

\begin{thBeispiel}[Vergissfunktor auf \texorpdfstring{$\Top$}{Top}]
    Sei $U\colon\Top\to\Set$ der Vergissfunktor, der einem topologischen Raum
    seine unterliegende Menge zuordnet. Seien weiter $D,K\colon\Set\to\Top$ die
    Funktoren, die eine Menge mit der diskreten bzw. der
    Klumpentopologie\footnote{auch indiskrete oder chaotische Topologie genannt}
    ausstatten und so zu einem topologischen Raum machen. Alle drei Funktoren
    bilden Morphismen auf sich selbst\footnote{streng genommen: auf dieselbe
    Abbildung als Morphismus in der jeweils anderen Kategorie} ab.
    Sei $(Y,T)$ ein topologischer Raum und seien $(X,T_D)$ bzw. $(Z,T_K)$
    topologische Räume mit der diskreten bzw. der Klumpentopologie. Dann gilt
    \[  \Top\bigl( (X,T_D), (Y,T) \bigr) = \Set(X,Y)  \qundq
        \Top\bigl( (Y,T), (Z,T_D) \bigr) = \Set(X,Y)
    , \]
    denn jede Abbildung aus einem Raum mit der diskreten Topologie und jede
    Abbildung in einen Raum mit der Klumpentopologie ist stetig. Dies zeigt,
    dass die Funktoren $D,K$ auf Morphismen wohldefiniert sind. Wir behaupten
    nun, dass wir eine Kette von Adjunktionen $D\leftadjoint U\leftadjoint K$
    haben. Dies lässt leicht überprüfen, denn es gilt für alle Mengen~$Y$ und
    alle topologischen Räume $(X,T)$:
    \begin{alignat*}{2}
        \Top\bigl( DY, (X,T) \bigr) &= \Set(Y,X) &&= \Set\bigl( Y, U(X,T)\bigr)
        \quad\text{und}\\
        \Set\bigl( U(X,T), Y \bigr) &= \Set(X,Y) &&= \Top\bigl( (X,T), KY \bigr)
    . \end{alignat*}
    Also liefern $\phi^D_{Y,(X,T)} \defeq \id_{\Set(Y,X)}$ und 
    $\phi^K_{(X,T),Y} \defeq \id_{\Set(X,Y)}$ Adjunktionen $(D,U,\phi^D)$
    und $(U,K,\phi^K)$ (-- die Natürlichkeit ist klar).
\end{thBeispiel}

\pagebreak[2]
\begin{thBeispiel}[Initiale und terminale Objekte]
    \label{ch1:bsp:initterm}
    Sei $\catA$ eine Kategorie und sei $\ast$ das einzige Objekt der
    Kategorie~$\One$.\footnote{Kategorie mit genau einem Objekt und genau einem
    Morphismus} Sei weiter $C$ der eindeutig bestimmte Funktor $\catA\to\One$.
    Dann besitzt $\catA$ genau dann ein terminales Objekt, wenn es einen
    zu $C$ rechtsadjungierten Funktor gibt, und $\catA$ besitzt genau dann ein
    initiales Objekt, wenn es einen zu $C$ linksadjungierten Funktor gibt.

    \newcommand{\cA}{\circledchar[black!40]{$A$}}
    \newcommand{\cX}{\circledchar[black!40]{$X$}}
    \newcommand{\cY}{\circledchar[black!40]{$Y$}}
    Wir bemerken zunächst, dass ein Funktor $\One\to\catA$ nichts weiter ist
    als ein Objekt in $\catA$. Daher schreiben wir im Folgenden einfach
    $\cA$ für den Funktor $\One\to\catA$, der $\ast$ auf das Objekt~$A$ aus
    $\catA$ abbildet.
    
    Sei $\cX\colon\One\to\catA$ ein Funktor mit $C\leftadjoint\cX$. Ist dann
    $A\in\catA$ ein Objekt aus $\catA$, so gilt:
    \[ \One(\ast,\ast) = \One(C\!A,\ast) \cong \catA(A,\cX\ast) = \catA(A,X) . \]
    Da $\One(\ast,\ast) = \{\id_\ast\}$ genau aus dem Identitätsmorphismus auf
    $\ast$ besteht, muss es also genau ein Element in $\catA(A,X)$ geben, d.\,h.
    es gibt genau einen Morphismus $A\to X$. Da $A\in\catA$ beliebig war, ist
    $X$ also ein terminales Objekt. Analog erhalten wir für einen Funktor
    $\cY\colon\One\to\catA$ mit $\cY\leftadjoint C$ für alle $A\in\catA$:
    \[ \catA(Y,A) = \catA(\cY\ast,A) \cong \One(\ast,C\!A) = \One(\ast,\ast) . \]
    Damit ist $Y$ ein initiales Objekt in $\catA$.

    Sei nun umgekehrt $X$ ein terminales Objekt in $\catA$. Dann gilt
    $C\leftadjoint\cX$ vermöge der Adjunktion
    $\bigl(C,\cX, \eta, \epsilon\bigr)$ mit 
    \[ \eta \defeq (A{\to}X)_{A\in\catA} \qqundqq \epsilon \defeq (\id_\ast)_\ast
    . \]
    Die Natürlichkeit von $\epsilon$ ist klar und die von $\eta$ prüfen wir
    schnell nach: Sei $f\colon A\to A'$ ein Morphismus in $\catA$. Dann müssen
    wir zeigen, dass das Diagramm 
    \vspace{-2mm}
    \[
        \xymatrix@=3.5mm{A \ar[r]^f \ar[d] & A' \ar[d] \\ X \ar[r] & X}
    \]
    kommutiert. Aber da $X$ ein terminales Objekt ist, bleibt diesem Diagramm
    gar nichts anderes übrig, als zu kommutieren. Wir rechnen außerdem nach,
    dass $\eta$ und $\epsilon$ die Dreiecksidentitäten erfüllen. 
    Sei $A\in\catA$. Dann gilt
    \[ (\epsilon C \after C\eta)_A 
        = \epsilon_\ast \after C(A{\to}X)
        = \id_\ast \after \id_\ast
        = \id_\ast = \id_{C\!A}
    \]
    sowie
    \[ (\cX\mkern1mu\epsilon \after \eta\mkern1mu\cX)_\ast
        = \cX(\id_\ast) \after \eta_X
        = \id_X \after (X{\to}X)
        = \id_X = \id_{\cX\ast}
    . \]
    Völlig analog zeigt man, dass für ein initiales Objekt $Y\in\catA$ das Tupel
    \[ \bigl( \cY,\; C,\; (\id_\ast)_\ast,\; (Y{\to}A)_{A\in\catA} \bigr) \]
    eine Adjunktion $\cY\leftadjoint C$ definiert.
\end{thBeispiel}


\nocite{lecnotes:leinster}
\nocite{bookc:maclane97}

\appendix
\bibliographystyle{plaindin}
\bibliography{bibsources}

\end{document}





